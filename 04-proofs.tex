\section{Properties of \CEU Reactions}
\label{sec.proofs}

In this section, we establish three properties of \CEU reactions.  First,
that reactions are deterministic.  Second, that they always produce a result
and terminate. And third, that the maximum amount of memory they consume
never exceeds a fixed threshold which can be derived statically from the
input program.

The proofs of determinism (Section~\ref{sec.proofs.det}) and termination
(Section~\ref{sec.proofs.term}) are constructed from the bottom up.  We
first prove the determinism and termination of single~$\out$ and~$\nst$
transitions, then of their finite iterations~$\out[*]$ and~$\nst[*]$, and
finally of the finite iteration of transitions in general
(relation~$\trans[*]$).  Since \emph{reaction} is formally defined in terms
of relation~$\trans[*]$ (see Section~\ref{sec.sem.opsem}), by proving
that~$\trans[*]$ is deterministic and that it always terminates (with an
irreducible description), we establish that both properties also hold for
\emph{reaction}.

The remaining property of reactions, memory-bound\-ed\-ness
(Section~\ref{sec.proofs.bound}), follows from termination and from the fact
that only a limited amount of expansions, declarations, and emissions can
occur during a reaction.

\subsection{Determinism}
\label{sec.proofs.det}

Transitions~$\out$ and~$\nst$ are defined in such a way that given an input
description either no rule is applicable or exactly one of them can be
applied (no choice involved).  This coupled with the fact that the output of
every rule is a function of its input implies that transitions are
deterministic: the same input description, if it can transition, will always
result in the same output description.  Thus the transition
relation~$\trans$ is in fact a (partial) function.

The next two lemmas establish the determinism of a single application
of~$\out$ and~$\nst$.  Lemma~\ref{lem.det-out} follows from the
inspection of rules~\R{push} and~\R{pop}.  The proof of
Lemma~\ref{lem.det-nst} is by induction on the structure of the derivation
trees produced by the rules for~$\nst$.  Both lemmas are used in the proof
of the Theorem~\ref{thm.det}, the main result of this section.

\begin{lemma}[label=lem.det-out,restate=lemdetout]
  If~$\delta\out\delta_1$ and~$\delta\out\delta_2$ then~$\delta_1=\delta_2$.
\end{lemma}

\begin{lemma}[label=lem.det-nst,restate=lemdetnst]
  If~$\delta\nst\delta_1$ and~$\delta\nst\delta_2$ then~$\delta_1=\delta_2$.
\end{lemma}

Theorem~\ref{thm.det} establishes that any number~$i\ge0$ of applications
of arbitrary transition rules, starting from the same input description,
will always lead to the same output description.  In other words, any finite
sequence of transitions behave deterministically.

\begin{theorem}[name=Determinism,label=thm.det,restate=thmdet]
  $\delta\trans[i]\delta_1$ and~$\delta\trans[i]\delta_2$
  implies~$\delta_1=\delta_2$.
\end{theorem}
\begin{proof}
  By induction on~$i$.  The theorem is trivially true if~$i=0$ and follows
  directly from the previous lemmas if~$i=1$.  Suppose
  \[
    \delta\trans[1]\delta_1'\trans[i-1]\delta_1
    \quad\text{and}\quad
    \delta\trans[1]\delta_2'\trans[i-1]\delta_2\,,
  \]
  for some~$i>1$, $\delta_1'$, and~$\delta_2'$.
  %%
  Then, by Lemma~\ref{lem.det-out} or~\ref{lem.det-nst}, depending on
  whether the first transition is~$\out$ or~$\nst$ (it cannot be both),
  $\delta_1'=\delta_2'$, and by the induction hypothesis,
  $\delta_1=\delta_2$.
\end{proof}

\subsection{Termination}
\label{sec.proofs.term}

We now turn to the problem of termination.  We want to show that any
sufficiently long sequence of applications of arbitrary transition rules
will eventually lead to an irreducible description, i.e., one that cannot be
modified by further transitions.  Before doing that, however, we need to
introduce some notation and establish some basic properties of the
transition relations~$\nst$ and~$\out$.

\begin{definition}[label=def.Hnst]
  A description~$\delta=\<\stmt,\ell,e,\theta>$ is
  \emph{nested-irre\-ducible} if, and only if, $e\ne\nil$
  or~$\stmt=\ceu{\Nop}$ or~$\stmt=\ceu{\Break}$ or~$\isblk(\stmt,\ell)$.
\end{definition}

Nested-irreducible descriptions serve as normal forms for~$\nst$
transitions: they embody the result of an exhaustive number of~$\nst$
applications.  We will write~$\delta_\Hnst$ to indicate that
description~$\delta$ is nested-irreducible.

The use of word ``irreducible'' in the previous definition is justified
by Proposition~\ref{prop.irr-nst-i}, which states that if a finite number
of applications of~$\nst$ results in an irreducible description, then that
occurs exactly once, at some specific number~$i$.  Its proof follows
directly from the definition of~$\nst$ by contradiction on the hypothesis
that there is such~$k\ne{i}$.

\begin{proposition}[label=prop.irr-nst-i,restate=propirrnsti]
  If~$\delta\nst[i]\delta_\Hnst'$ then, for all~$k\ne{i}$, there is
  no~$\delta_\Hnst''$ such that~$\delta\nst[k]\delta''_\Hnst$.
\end{proposition}

The next lemma establishes that sequences of~$\nst$ transitions behave as
expected regarding the internal evaluation of compound statements.  Its
proof is by induction on~$i$.

\begin{lemma}[label=lem.props-nst-i,restate=lempropsnsti]
  \strut
  \begin{enumerate}
  \item\label{lem.props-nst-i.1}
    If~$\<\stmt_1,\nil,\theta>\nst[i]\<\stmt_1',e,\theta'>$ then, for
    any~$\stmt_2$,
    \begin{align*}
      \begin{split}
        &\<\ceu{\stmt_1;\,\stmt_2},\nil,\theta>\\
        &\qquad\nst[i]\<\stmt_1';\stmt_2,e,\theta'>,
      \end{split}\\
      %% 
      \begin{split}
        &\<\ceu{\stmt_1\AtLoop{p_2}},\stmt_2,\nil,\theta>\\
        &\qquad\nst[i]\<\ceu{\stmt_1'\AtLoop{\stmt_2}},e,\theta'>,
      \end{split}\\
      %% 
      \begin{split}
        &\<\ceu{{\stmt_1}\AtParAnd{\stmt_2}},\nil,\theta>\\
        &\qquad\nst[i]\<\ceu{{\stmt_1'}\AtParAnd{\stmt_2}},e,\theta'>,
      \end{split}\\
      %% 
      \begin{split}
        &\<\ceu{{\stmt_1}\AtParOr{\stmt_2}},\nil,\theta>\\
        &\qquad\nst[i]\<\ceu{{\stmt_1}'\AtParOr{\stmt_2}},e,\theta'>.
      \end{split}
    \end{align*}
    %%
  \item If~$\<\stmt_2,\ell,\nil,\theta>\nst[i]\<\stmt_2',\ell,e,\theta'>$
    then, for any~$\stmt_1$ such that~$\isblk(\stmt_1,\ell)$,
    \begin{align*}
      \begin{split}
        &\<\ceu{\stmt_1\AtParAnd\stmt_2},\nil,\theta>\\
        &\qquad\nst[i]\<\ceu{\stmt_1\AtParAnd\stmt_2'},e,\theta'>,
      \end{split}\\
      %% 
      \begin{split}
        &\<\ceu{\stmt_1\AtParOr\stmt_2},\nil,\theta>\\
        &\qquad\nst[i]\<\ceu{\stmt_1\AtParOr\stmt_2'},e,\theta'>.
      \end{split}
    \end{align*}
  \end{enumerate}
\end{lemma}

The syntactic restrictions presented in Section~\ref{sec.sem.restrictions}
regarding the bodies of loops, $\ceu{\Every}$ statements, and $\ceu{\Fin}$
statements give rise to the following propositions.

\begin{proposition}[label=prop.syn-rest.loop,restate=propsynrestloop]
  If~$\stmt$ is the body of a $\ceu{\Loop}$, then
  \[
    \<\stmt,\ell,\nil,\theta>\nst[*]\<\stmt',\ell,e,\theta'>,
  \]
  where $\stmt'=\ceu{\Break\AtLoop\stmt}$ or $\isblk(\stmt',\ell)$.
\end{proposition}

\begin{proposition}[label=prop.syn-rest.fin,restate=propsynrestfin]
  If~$\stmt$ is the body of an $\ceu{\Every}$ or~$\ceu{\Fin}$, then
  \[
    \<\clear(\stmt),\ell,\nil,\theta>\nst[*]\<\stmt',\ell,\nil,\theta'>,
  \]
  where $\stmt'=\ceu{\Nop}$ or $\isblk(\stmt',\ell)$.
\end{proposition}

The next theorem, Theorem~\ref{thm.term-nst-*}, establishes that a finite
(possibly zero) number of~$\nst$ transitions eventually leads to a
nested-irreducible description.  Hence, for any input description~$\delta$,
it is always possible to transform~$\delta$ in a nested-irreducible
description~$\delta'$ by applying to it a sufficiently long sequence
of~$\nst$ transitions.  The proof of the theorem follows by induction on the
structure of statements and depends on Lemma~\ref{lem.props-nst-i} and
Propositions~\ref{prop.syn-rest.loop} and~\ref{prop.syn-rest.fin}.

\begin{theorem}[label=thm.term-nst-*,restate=thmtermnstx]
  For any~$\delta$ there is a~$\delta'_\Hnst$ such
  that~$\delta\nst[*]\delta'_\Hnst$.
\end{theorem}

The main result of this section, Theorem~\ref{thm.term}, is similar to
Theorem~\ref{thm.term-nst-*} but applies to transitions~$\trans$ in
general.  In the statement of Theorem~\ref{thm.term-nst-*} we use the
notion of a general irreducible description~$\delta_\#$, i.e., one that
cannot be advanced by $\out$ and $\nst$.  This notion is defined with the
help of the auxiliary notions of potency and rank, introduced below.

\begin{definition}
  \label{def.pot}
  %%
  The \emph{potency} of a statement $\stmt$ in reaction to event~$e$,
  denoted~$\pot(\stmt,e)$, is the maximum number of~$\ceu{\EmitInt}$
  statements that can be executed in a reaction of~$\stmt$ to~$e$, i.e.,
  \[
    \pot(\stmt,e)=\pot'(\bcast(\stmt,e))\,,
  \]
  where~$\pot'$ is a function that counts the maximum number of
  reachable~$\ceu{\EmitInt}$'s in the given statement (see
  Figure~\ref{fig.pot'}).
\end{definition}

\begin{figure*}[t]
  \newsavebox{\tmpbox}
  \sbox{\tmpbox}{$\pot'(\stmt_1)+\pot'(\stmt_2)$}
  \newlength{\tmplen}
  \setlength{\tmplen}{.75\columnwidth}
  \begin{minipage}{\textwidth}
    \begin{empheq}[box=\fbox]{alignat*=1}
      \pot'(\ceu{\EmitInt(e)})
      &=1\\
      %%
      \pot'(\ceu{\Var{v\,\stmt}})
      &=\pot'(\stmt)\\
      %%
      \pot'(\ceu{\IfElse{\expr}{\stmt_1}{\stmt_2}})
      &=\max\{\pot'(\stmt_1),\pot'(\stmt_2)\}\\
      %%
      \pot'(\ceu{\stmt_1;\stmt_2})
      &=\begin{cases}
        0
        &\begin{minipage}[t]{\tmplen}
          if $\stmt_1=\ceu{\Break}$ or $\stmt_1=\ceu{\AwaitExt}$\\
          \strut\quad or $\stmt_1=\ceu{\EveryExt}$
        \end{minipage}\\
        \pot'(\stmt_1)+\pot'(\stmt_2)
        &\text{otherwise}
      \end{cases}\\
      %%
      \pot'(\ceu{\Loop\stmt})
      &=\pot'(\stmt)\\
      %%
      \pot'(\ceu{\Every{e\,\stmt}})
      &=\begin{cases}
        0
        &\text{if~$e$ is an external event}\\
        \mathmakebox[\wd\tmpbox][l]{\pot'(\stmt)}
        &\text{otherwise}
      \end{cases}\\
      %%
      \pot'(\ceu{\stmt_1\ParAnd\stmt_2})
      &=\pot'(\stmt_1)+\pot'(\stmt_2)\\
      %%
      \pot'(\ceu{\stmt_1\ParOr\stmt_2})
      &=\pot'(\stmt_1)+\pot'(\stmt_2)\\
      %%
      \pot'(\ceu{\Fin\stmt})
      &=\pot'(\stmt)\\
      %%
      \pot'(\ceu{\AtVar{v\,n\,\stmt}})
      &=\pot'(\stmt)\\
      %%
      \pot'(\ceu{\stmt_1\AtLoop\stmt_2})
      &=
      \begin{cases}
        \pot'(\stmt_1)
        &\begin{minipage}[t]{\tmplen}
          if all execution paths in~$\stmt_1$ have a\\
          \strut\quad matching
          $\ceu{\Break}$ or $\ceu{\AwaitExt}$ or $\ceu{\EveryExt}$
        \end{minipage}\\
        %%
        \pot'(\stmt_1)+\pot'(\stmt_2)
        &\text{otherwise}
      \end{cases}\\
      %%
      \pot'(\ceu{\stmt_1\AtParAnd\stmt_2})
      &=
      \begin{cases}
        0                             &\text{if~$\stmt_1=\ceu{\Break}$}\\
        \pot'(\stmt_1)+\pot'(\stmt_2) &\text{otherwise}
      \end{cases}\\
      %%
      \pot'(\ceu{\stmt_1\AtParOr\stmt_2})
      &=
      \begin{cases}
        0
        &\text{if~$\stmt_1=\ceu{\Nop}$ or $\stmt_1=\ceu{\Break}$}\\
        \pot'(\stmt_1)+\pot'(\stmt_2)
        &\text{otherwise}
      \end{cases}\\
      %%
      \pot'(\_)
      &= 0\quad\text{\emph{(otherwise)}}
    \end{empheq}
  \end{minipage}
  \caption{Function~$\pot'$ counts the maximum number of
    reachable~$\ceu{\protect\EmitInt}$'s in the given statement.}
  \label{fig.pot'}
\end{figure*}

\begin{definition}
  \label{def.rank}
  %%
  The \emph{rank} of a description~$\delta=\<\stmt,\ell,e,\theta>$,
  denoted~$\rank(\delta)$, is a pair~$(i,j)$ of nonnegative integers where
  \begin{alignat*}{2}
    i&=\pot(\stmt,e) &\quad\text{and}\quad
    j&=
       \begin{cases}
         \ell  &\text{if~$e=\nil$}\\
         \ell+1&\text{otherwise\,.}
       \end{cases}
  \end{alignat*}
\end{definition}

Intuitively, the rank of a description~$\delta$ is a measure of the effort
required to transform~$\delta$ into an irreducible description, i.e., into
one that satisfies the next definition.

\begin{definition}
  \label{def.H}
  %%
  A description~$\delta$ is \emph{irreducible} (in symbols, $\delta_\#$) if,
  and only if, it is nested-irreducible and its~$\rank(\delta)$ is~$(i,0)$,
  for some~$i\ge0$.
\end{definition}

An irreducible description~$\delta_\#=\<\stmt,\ell,e,\theta>$ serves as a
normal form for transitions~$\trans$ in general.  Such description cannot be
advanced by~$\nst$, as it is nested-irreducible, and neither by~$\outpush$
nor~$\outpop$, as the second coordinate of its rank is~0, which
implies~$e=\nil$ and~$\ell=0$.

The next two lemmas establish that a single application of~$\out$ or~$\nst$
either preserves or decreases the rank of the input description.  All rank
comparisons use the lexicographic order, i.e., if~$\rank(\delta)=(i,j)$
and~$\rank(\delta')=(i',j')$ then
\[
  \rank(\delta)>\rank(\delta')
\]
if, and only if,
\[
  i>i'\quad\text{or}\quad\text{($i=i$ and $j>j'$)}.
\]

The proof of Lemma~\ref{lem.rank-out} follows directly from~\R{push}
and~\R{pop} and from the definitions of $\pot$ and $\rank$.  The proof of
Lemma~\ref{lem.rank-nst}, however, is by induction on the structure
of~$\nst$ derivations.

\begin{lemma}[label=lem.rank-out,restate=lemrankout]
  \strut
  %%
  \begin{enumerate}
  \item\label{lem.rank-out-push} If~$\delta\outpush\delta'$
    then~$\rank(\delta)=\rank(\delta')$.
  \item\label{lem.rank-out-pop} If~$\delta\outpop\delta'$
    then~$\rank(\delta)>\rank(\delta')$.
  \end{enumerate}
\end{lemma}

\begin{lemma}[label=lem.rank-nst,restate=lemranknst]
  If~$\delta\nst\delta'$ then~$\rank(\delta)\ge\rank(\delta')$.
\end{lemma}

The next theorem is a generalization of Lemma~\ref{lem.rank-nst}
for~$\nst[*]$.  Its proof follows from the lemma by induction on~$i$.

\begin{theorem}[label=thm.rank-nst-*,restate=thmranknstx]
  If~$\delta\nst[*]\delta'$ then~$\rank(\delta)\ge\rank(\delta')$.
\end{theorem}

We are finally in a position to state and prove the main result of this
section, Theorem~\ref{thm.term}, which is the termination theorem
for~$\trans[*]$.  The idea of the proof is that a sufficiently large
sequence of~$\nst$ and~$\out$ transitions eventually decreases the rank of
the current description until an irreducible description is reached.  This
irreducible description is the final result of the reaction.
%%
\begin{restatable}[Termination]{theorem}{thmterm}
  \label{thm.term}
  %%
  For any~$\delta$, there is a~$\delta'_\#$ such
  that~$\delta\trans[*]\delta'_\#$.
\end{restatable}
\begin{proof}
  By induction on induction on~$\rank(\delta)$.
  Let
  \[
    \delta=\<\stmt,\ell,e,\theta>
    \quad\text{and}\quad
    \rank(\delta)=(i,j).
  \]

  For the basis, suppose $(i,j)=(0,0)$.  Then~$\delta$ cannot be advanced
  by~$\out$, as~$j=0$ implies~$e=\nil$ and~$\ell=0$.  If~$\delta$ is
  nested-irreducible, the theorem is trivially true,
  as~$\delta\nst[0]\delta_\Hnst$ and~$\delta_\#$.  If~$\delta$ is not
  nested-irreducible then, by Theorem~\ref{thm.term-nst-*},
  $\delta\nst[*]\delta'_\Hnst$, for some~$\delta'_\Hnst$.  By
  Theorem~\ref{thm.rank-nst-*}, $\rank(\delta)\ge\rank(\delta')$, which
  implies~$\rank(\delta')=(0,0)$, and so~$\delta'_\#$.

  For the inductive step, suppose~$(i,j)>(0,0)$.  There are two
  cases.
  \begin{case}
  \item[{$[\delta\text{~is nested-irreducible}]$}]\label{thm.term.Hnst}
    %%
    If~$j=0$, by Definition~\ref{def.H}, $\delta_\#$, and
    so~$\delta\trans[0]\delta_\#$.  If~$j>0$, there are two subcases:
    \begin{case}
    \item[{[$e\ne\nil$]}]\label{thm.term.Hnst-j>0-nonnil}
      %%
      Then, by~\R{push} and by Theorem~\ref{thm.term-nst-*}, there
      are~$\delta_1'$ and~$\delta'_\Hnst=\<\stmt',\ell+1,e',\theta'>$ such
      that
      \[
        \delta\outpush\delta_1'\nst[*]\delta'_\Hnst.
      \]
      Thus, by Lemma~\ref{lem.rank-out} and by
      Theorem~\ref{thm.rank-nst-*},
      \begin{align*}
        \rank(\delta)=\rank(\delta_1')=(i,j)
        \ge\rank(\delta')=(i',j').
      \end{align*}
      If~$e'=\nil$, then $i=i'$ and~$j=j'$, and the rest of this proof is
      similar to that of the case for $e=\nil$ below.  Otherwise,
      if~$e'\ne\nil$, then $i>i'$, as at least one reachable
      $\ceu{\EmitInt}$ was consumed by the nested transitions.  Thus,
      $\rank(\delta)>\rank(\delta')$.  By the induction hypothesis,
      $\delta'\trans[*]\delta''_\#$, for some~$\delta''_\#$.  Therefore,
      $\delta\trans[*]\delta''_\#$.
      %%
    \item[{[$e=\nil$]}]\label{thm.term.Hnst-j>0-nil}
      %%
      Then, as~$j>0$, $\delta\outpop\delta'$, for some~$\delta'$.  By
      Lemma~\ref{lem.rank-out}, $\rank(\delta)>\rank(\delta')$.  Hence, by
      the induction hypothesis, $\delta'\trans[*]\delta''_\#$, for
      some~$\delta''_\#$.  And so, $\delta\trans[*]\delta''_\#$.
    \end{case}
    %%
  \item[{$[\delta\text{~is not nested-irreducible}]$}]
    %%
    Then~$e=\nil$ and, by Theorems~\ref{thm.term-nst-*}
    and~\ref{thm.rank-nst-*}, there is a~$\delta'_\Hnst$ such
    that~$\delta\nst[*]\delta'_\Hnst$
    with~$\rank(\delta)\ge\rank(\delta'_\Hnst)$.  The rest of this proof is
    similar to that of case when $\delta$ is nested-irreducible
    above.\qedhere
  \end{case}
\end{proof}

\subsection{Memory Boundedness}
\label{sec.proofs.bound}

\CEU has no mechanism for heap allocation, unbounded iteration, or general
recursion.  Consequently, the memory (or space) usage of a given \CEU
program never exceeds a fixed threshold, which depends solely on the
program's text.  We can define this threshold for a single reaction as
follows.

The maximum space used by the \CEU interpreter to compute a reaction of
program~$\stmt$ to external event~$e$, denoted $\reactspc(\stmt,e)$,
corresponds to the maximum space it uses when computing
\begin{equation}
  \label{eq.bound}
  \<\stmt,0,e,[]>\trans[*]\<\stmt',0,\nil,[]>_\#.
\end{equation}

If we assume that to compute~\eqref{eq.bound} the interpreter only keeps
information about the current description~$\delta_c$, and that recursive
$\nst$ transitions are computed inline (by rewriting~$\delta_c$), then
$\reactspc(\stmt,e)$ is the maximum space taken by:
\begin{enumerate}
\item the statement part~$\stmt_c$ of~$\delta_c$;
\item the memory part~$\theta_c$ of~$\delta_c$; and
\item stack level part~$\ell_c$ of~$\delta_c$.
\end{enumerate}
%%
The three objects, $\stmt_c$, $\theta_c$, and $\ell_c$, can only grow in
proportion to the part of~$\stmt_c$ that can be advanced during the
reaction, which we will call \emph{reachable part} of $\stmt_c$.

The maximum size for~$\stmt_c$ is linearly proportional to the maximum
number of \textbf{$\ast$-expd} applications that can occur in the reachable
part of $\stmt$ in~\eqref{eq.bound}.  These expansions are the only rule
applications that increase the size of~$\stmt_c$.

The maximum size for~$\theta_c$ is linearly proportional to the maximum
number of variables simultaneously declared (in the same scope) in the
reachable part of $\stmt$ in~\eqref{eq.bound}.  Each of these declarations,
if advanced, will cause~$\theta_c$ to grow by a constant.

Finally, the maximum value for~$\ell_c$ is equal to the maximum number of
$\out$ transitions that can occur in~\eqref{eq.bound}, which in turn is
equal to the maximum number of $\ceu{\EmitInt}$'s that can be consumed in
the reachable part of $\stmt$ in~\eqref{eq.bound}, i.e., $\pot(\stmt,e)$,
see $\pot$ in Section~\ref{sec.proofs.term}.

We can extend the definition of maximal space to an arbitrary reaction as
follows.  Let~$\stmt$ be a program that can react to external events~$e_1$,
\dots,~$e_n$ then the maximal space taken by any reaction of~$\stmt$ is
\[
  \max\{\reactspc(\stmt,e_1),\dots,\reactspc(\stmt,e_n)\}.
\]

Similarly, we can extend the definition of maximal space to histories of
external events (or sequences of reactions), and lastly, to the worst-case
prefix of such a history, i.e., to the prefix which would trigger the worst
possible $\reactspc$ call.  But this is out of the scope of this paper.

% - program is finite
% - lexical scope
%     - no heap allocation
% - no code reentrancy
%     - reexecution only due to loops
%     - loop reuse nested vars

%%% Local Variables:
%%% mode: latex
%%% TeX-master: "index.tex"
%%% End:
