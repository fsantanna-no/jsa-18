\section{Properties of \CEU Programs}
\label{sec.proofs}

\subsection{Deterministic Execution}

Transitions~$\out$ and~$\nst$ are defined in such a way that given an input
description either no rule is applicable or exactly one of them can be
applied (no choice involved).  This coupled with the fact that the output of
every rule is a function of its input implies that transitions are
deterministic: the same input description, if it can transition, will always
result in the same output description.  Thus the transition
relation~$\trans$ is in fact a partial function.

The next two lemmas establish the determinism of a single application
of~$\out$ and~$\nst$.  Lemma~\ref{lem.x.det-out} follows from a simple
inspection of rules~\R{push} and~\R{pop}.  The proof of
Lemma~\ref{lem.x.det-nst} follows by induction on the structure of the
derivation trees produced by the rules for~$\nst$.  Both lemmas are used in
the proof of the Theorem~\ref{thm.x.det}.

\begin{restatable}{lemma}{lemxdetout}
  \label{lem.x.det-out}
  %%
  If~$\delta\out\delta_1$ and~$\delta\out\delta_2$ then~$\delta_1=\delta_2$.
\end{restatable}

\begin{restatable}{lemma}{lemxdetnst}
  \label{lem.x.det-nst}
  %%
  If~$\delta\nst\delta_1$ and~$\delta\nst\delta_2$ then~$\delta_1=\delta_2$.
\end{restatable}

The main result of this section, Theorem~\ref{thm.x.det}, establishes that
any given number~$i\ge0$ of applications of arbitrary transition rules,
starting from the same input description, will always lead to the same
output description.  In other words, any finite sequence of transitions
behave deterministically.

\begin{restatable}[Determinism]{theorem}{thmxdet}
  \label{thm.x.det}
  %%
  $\delta\trans[i]\delta_1$ and~$\delta\trans[i]\delta_2$
  implies~$\delta_1=\delta_2$.
\end{restatable}
\begin{proof}
  By induction on~$i$.  The theorem is trivially true if~$i=0$ and follows
  directly from the previous lemmas if~$i=1$.  Suppose
  \[
    \delta\trans[1]\delta_1'\trans[i-1]\delta_1
    \quad\text{and}\quad
    \delta\trans[1]\delta_2'\trans[i-1]\delta_2\,,
  \]
  for some~$i>1$, $\delta_1'$ and~$\delta_2'$.
  %%
  Then, by Lemma~\ref{lem.x.det-out} or~\ref{lem.x.det-nst}, depending on
  whether the first transition is~$\out$ or~$\nst$ (it cannot be both),
  $\delta_1'=\delta_2'$, and by the induction hypothesis,
  $\delta_1=\delta_2$.
\end{proof}

% The proof for determinism relies on the fact all semantic rules are
% mutually exclusive, i.e., their preconditions are unique in the set of
% rules.  This can be verified by direct inspection of rules.

% Rule \textbf{push} is the only one with $e \neq \epsilon$ as a
% precondition, and is trivially mutually exclusive with all other rules.

% Rule \textbf{pop} either has $p=@nop$ or $isblocked(p,n)$ as
% preconditions.
% %
% Note that rule \textbf{pop} only applies syntactically to top-level
% transitions.  For instance, it can never match $\NST$ rules for
% subprograms as in rule \textbf{seq-adv}.
% %
% Hence, for the first case, rule \textbf{pop} only applies, and is the only
% one to apply, to $nop$ as the whole program (i.e., a $nop$ not surrounded
% by other expressions, such as in rule \textbf{seq-nop}).
% %
% For the second case, we need to show that given $\LL p,n,\epsilon \RR$, no
% $\NST$ transitions apply with $isblocked(p,n)$ and vice versa.  Except for
% $@canrun$, there are no $\NST$ transitions for the other blocking
% expressions ($awaitExt$, $awaitInt$, $every$, and $fin$).  However,
% considering the precondition $\LL p,n,\epsilon \RR$,
% $isblocked(@canrun(n),n)$ is false.  Hence, given the preconditions for
% rule \textbf{pop}, no $\NST$ transitions can occur.  Conversely, if a
% $\NST$ transition is possible, then $isblocked(p,n)$ must be false.
% Again, except for $@canrun$, all other transitions do not involve blocking
% expressions, hence, for these transitions, $isblocked(p,n)$ must be false.
% For rule \R{can-run}, a transition can only occur if the current stack
% level matches $@canrun(n)$.  In this case, $isblocked(@canrun(n),n)$ is
% false.

% Finally, we need to show that $\NST$ transitions are mutually exclusive
% among themselves.
% %
% Note that most rules have unique syntactic prefixes, e.g., $(@nop~@and~q)$
% (rule \textbf{and-nop1}) is trivially mutually exclusive with
% $(@nop~@loop~p)$ (rule \textbf{or-nop1}).
% %
% The only exceptions are rules \textbf{and-adv1} vs. \textbf{and-adv2}, and
% \textbf{or-adv1} vs. \textbf{or-adv2}.  In both cases, we need to show
% that if the left branch can advance, then it cannot be blocked and
% vice-versa, i.e., that $\LL p,n,\epsilon \RR \NST \LL p',n,e \RR$ and
% $isblocked(p,n)$ are mutually exclusive, which is exactly the same
% reasoning for rule \textbf{pop} above.

\subsection{Terminating Reactions}

We now turn to the problem of termination.  We want to show that any
sufficiently long sequence of applications of arbitrary transition rules
will eventually lead to an irreducible description, i.e., one that cannot be
modified by further transitions.  Before doing that, however, we need to
introduce some notation and establish some basic properties of
the transition relations~$\nst$ and~$\out$.

\begin{definition}
  \label{def.x.Hnst}
  %%
  A description~$\delta=\<p,n,e>$ is \emph{nested-irre\-ducible}
  iff~$e\ne\nil$ or~$p=\ceu{\Nop}$ or~$p=\ceu{\Break}$
  or~$\isblocked(p,n)$.\footnote{We sometimes abbreviate ``$p=\ceu{\Nop}$
    or~$p=\ceu{\Break}$'' as~``$p=\ceu{\Nop,\Break}$''.}
\end{definition}

Nested-irreducible descriptions serve as normal forms for~$\nst$
transitions: they embody the result of an exhaustive number of~$\nst$
applications.  We will write~$\delta_\Hnst$ to indicate that
description~$\delta$ is nested-irreducible.

The use of qualifier ``irreducible'' in Definition~\ref{def.x.Hnst} is
justified by Proposition~\ref{prop.x.irr-nst-i}, which states that if a
finite number of applications of~$\nst$ results in an irreducible
description, then that occurs exactly once, at some specific number~$i$.
The proof of Proposition~\ref{prop.x.irr-nst-i} follows directly from the
definition of~$\nst$ by contradiction on the hypothesis that there is
such~$k\ne{i}$.

\begin{restatable}{proposition}{propxirrnsti}
  \label{prop.x.irr-nst-i}
  %%
  If~$\delta\nst[i]\delta_\Hnst'$ then, for all~$k\ne{i}$, there is
  no~$\delta_\Hnst''$ such that~$\delta\nst[k]\delta''_\Hnst$.
\end{restatable}

The next lemma establishes that sequences of~$\nst$ transitions behave as
expected regarding the order of evaluation of composition branches.  Its
proof follows by induction on~$i$.

\begin{restatable}{lemma}{lemxpropsnsti}
  \label{lem.x.props-nst-i}\strut\\
  %%
  If~$\<p_1,n,e>\nst[i]\<p_1',n,e'>$, for any~$p_2$:
  \begin{enumerate:a}
  \item\label{lem.x.props-nst-i.a}
    $\<\ceu{p_1;\,p_2},n,e>\nst[i]\<p_1';p_2,n,e'>$.
    %%
  \item\label{lem.x.props-nst-i.b}
    $\<\ceu{p_1\AtLoop{p_2}},n,e>\nst[i]\<\ceu{p_1'\AtLoop{p_2}},n,e'>$.
    %%
  \item\label{lem.x.props-nst-i.c}
    $\<\ceu{{p_1}\AtAnd{p_2}},n,e>\nst[i]\<\ceu{{p_1'}\AtAnd{p_2}},n,e'>$.
    %%
  \item\label{lem.x.props-nst-i.d}
    $\<\ceu{{p_1}\AtOr{p_2}},n,e>\nst[i]\<\ceu{{p_1}'\AtOr{p_2}},n,e'>$.
  \end{enumerate:a}
  \smallskip
  If~$\<p_2,n,e>\nst[i]\<p_2',n,e'>$, for any~$p_1$ such
  that~$\isblocked(p_1,n)$:
  \begin{enumerate:a}
    \setcounter{enumi}{4}
  \item\label{lem.x.props-nst-i.e}
    $\<\ceu{{p_1}\AtAnd{p_2}},n,e>\nst[i]\<\ceu{{p_1}\AtAnd{p_2'}},n,e'>$.
    %%
  \item\label{lem.x.props-nst-i.f}
    $\<\ceu{{p_1}\AtOr{p_2}},n,e>\nst[i]\<\ceu{{p_1}\AtOr{p_2'}},n,e'>$.
  \end{enumerate:a}
\end{restatable}
%%
% \begin{proof}[Proof of~(a)]
%   By induction on~$i$.  The lemma is trivially true for~$i=0$,
%   as~$p_1=p_1'$, and follows directly from~\R{seq-adv} for~$i=1$.  Suppose
%   \begin{equation}
%     \label{lem.x.props-nst-i.a.eq1}
%     \<p_1,n,e>\nst[1]\<p_1'',n,e''>\nst[i-1]\<p_1',n,e'>\,,
%   \end{equation}
%   for some~$i>1$.  Then~$\<p_1'',n,e''>$ is not nested-irreducible, i.e.,
%   $e=\nil$ and~$p\ne{\ceu{\Nop},\ceu{\Break}}$ and~$\isblocked(p_1'',n)$
%   is false.  By~\eqref{lem.x.props-nst-i.a.eq1} and by~\R{seq-adv},
%   \begin{equation}
%     \label{lem.x.props-nst-i.a.eq2}
%     \<\ceu{p_1;\,p_2},n,e>\nst[1]\<\ceu{p_1'';\,p_2},n,e''>\,.
%   \end{equation}
%   From~\eqref{lem.x.props-nst-i.a.eq1}, by the induction hypothesis,
%   \begin{equation}
%     \label{lem.x.props-nst-i.a.eq3}
%     \<\ceu{p_1'';\,p_2},n,e''>\nst[i-1]\<\ceu{p_1';\,p_2},n,e'>\,.
%   \end{equation}
%   From~\eqref{lem.x.props-nst-i.a.eq2} and~\eqref{lem.x.props-nst-i.a.eq3},
%   \[
%     \<\ceu{p_1;\,p_2},n,e>\nst[i]\<\ceu{p_1';\,p_2},n,e'>\,.
%   \]
%   The proofs of the other cases are similar.
% \end{proof}
%%

The syntactic restriction discussed in Section~\ref{sec.ceu.evts} regarding
the body of loops and the restriction mentioned in
Section~\ref{sec.sem.opsem} about the body of~$\ceu{\Fin}$ statements are
formalized in Assumption~\ref{ass.x.syn-rest} below.  These restrictions are
essential to prove the next theorem.

\begin{assumption}[Syntactic restrictions]\strut
  \label{ass.x.syn-rest}
  %%
  \begin{enumerate:a}
  \item\label{ass.x.syn-rest.fin} If~$p=\ceu{\Fin{p_1}}$ then~$p_1$ contains
    no occurrences of statements~$\ceu{\AwaitExt}$, $\ceu{\AwaitInt}$,
    $\ceu{\Every}$, or~$\ceu{\Fin}$.  And so, for any~$n$,
    $\<\clear(p_1),n,\nil>\nst[*]\<\ceu{\Nop},n,\nil>$.
    %%
    % \[
    %   \<\clear(p_1),n,\nil>\nst[*]\<\ceu{\Nop},n,\nil>\,.
    % \]
    %%
  \item\label{ass.x.syn-rest.loop} If~$p=\ceu{\Loop{p_1}}$ then all
    execution paths of~$p_1$ contain a matching~$\ceu{\Break}$ or
    an~$\ceu{\AwaitExt}$.  Consequently, for all~$n$, there are~$p_1'$
    and~$e$ such that
    $\<\ceu{\Loop{p_1}},n,\nil>\nst[*]\<p_1',n,e>$,
    %%
    % \[
    %   \<\ceu{\Loop{p_1}},n,\nil>\nst[*]\<p_1',n,e>\,,
    % \]
    %%
    where $p_1'=\ceu{{\Break}\AtLoop{p_1}}$ or~$\isblocked(p_1',n)$\,.
  \end{enumerate:a}
\end{assumption}

Theorem~\ref{thm.x.term-nst-*} establishes that a finite (possibly zero)
number of~$\nst$ transitions eventually leads to a nested-irreducible
description.  Hence, for any input description~$\delta$, it is always
possible to transform~$\delta$ in a nested-irreducible description~$\delta'$
by applying to it a sufficiently long sequence of~$\nst$ transitions.  The
proof of the theorem follows by induction on the structure of programs
(members of set~$P$) and depends on Lemma~\ref{lem.x.props-nst-i} and
Assumption~\ref{ass.x.syn-rest}.

\begin{restatable}{theorem}{thmxtermnstx}
  \label{thm.x.term-nst-*}
  %%
  For any~$\delta$ there is a~$\delta'_\Hnst$ such
  that~$\delta\nst[*]\delta'_\Hnst$.
\end{restatable}

The main result of this section, Theorem~\ref{thm.x.term}, is similar to
Theorem~\ref{thm.x.term-nst-*} but applies to transitions~$\trans$ in
general.  Before stating and proving it, we need to characterize irreducible
descriptions in general.  This characterization, given in
Definition~\ref{def.x.H}, depends on the notions of potency and rank.

\begin{definition}
  \label{def.x.pot}
  %%
  The \emph{potency} of a program~$p$ in reaction to event~$e$,
  denoted~$\pot(p,e)$, is the maximum number of~$\ceu{\EmitInt}$ statements
  that can be executed in a reaction of~$p$ to~$e$, i.e.,
  \[
    \pot(p,e)=\pot'(\bcast(p,e))\,,
  \]
  where~$\pot'$ is an auxiliary function that counts the maximum number of
  reachable~$\ceu{\EmitInt}$ statements in the program resulting from the
  broadcast of event~$e$ to~$p$.

  Function~$\pot'$ is defined by the following clauses:
  \begin{enumerate:a}
  \item\label{def.x.pot.first}$\pot'(\ceu{\EmitInt}(e))=1$.
    %%
  \item$\pot'(\ceu{\IfElse{\Mem(\Id)\!}{p_1\!}{p_2\!}})\!=\!
    \max\{\pot'(p_1),\pot'(p_2)\}$.
    %%
  \item$\pot'(\ceu{\Loop{p_1}})=\pot'(p_1)$.
    %%
  \item$\pot'(\ceu{{p_1}\And{p_2}})=\pot'(\ceu{{p_1}\Or{p_2}})=
    \pot'(p_1)+\pot'(p_2)$.
    %%
  \item If~$p_1\ne\ceu{\Break},\ceu{\AwaitExt(e)}$,
    \begin{align*}
      \pot'(\ceu{p_1;\,p_2})&=\pot'(p_1)+\pot'(p_2)\\[-.5\jot]
      \pot'(\ceu{p_1\AtLoop{p_2}})&=
      \begin{cases}
        \pot'(p_1)              &\text{if\enspace(\dag)}\\[-.5\jot]
        \pot'(p_1)+\pot'(p_2)   &\text{otherwise,}\\[-.5\jot]
      \end{cases}
    \end{align*}
    where~(\dag) stands for: ``a~$\ceu{\Break}$ or~$\ceu{\AwaitExt}$ occurs
    in all execution paths of~$p_1$''.
    %%
  \item If~$p_1,p_2\ne\ceu{\Break}$,
    $\pot'(\ceu{{p_1}\AtAnd{p_2}})=\pot'(p_1)+\pot'(p_2)$.
    %%
  \item\label{def.x.pot.before-last} If~$p_1,p_2\ne\ceu{\Break}$
    and~$p_1,p_2\ne\ceu{\Nop}$,
    \vskip-.95em
    \[
      \pot'(\ceu{{p_1}\AtOr{p_2}})=\pot'(p_1)+\pot'(p_2)\,.
    \]
    %%
  \item\label{def.x.pot.last} Otherwise, if none
    of~\ref{def.x.pot.first}--\ref{def.x.pot.before-last} applies,
    $\pot'(\_)=0$.
  \end{enumerate:a}
\end{definition}

\begin{definition}
  \label{def.x.rank}
  %%
  The \emph{rank} of a description~$\delta=\<p,n,e>$,
  denoted~$\rank(\delta)$, is a pair of nonnegative integers~$\<i,j>$ such
  that
  \begin{alignat*}{2}
    i&=\pot(p,e) &\quad\text{and}\quad
    j&=
       \begin{cases}
         n  &\text{if~$e=\nil$}\\
         n+1&\text{otherwise\,.}
       \end{cases}
  \end{alignat*}
Intuitively, the rank of a description~$\delta$ is a measure of the maximum
amount of ``work'' (transitions) required to transform~$\delta$ into an
irreducible description, in the following sense.
\end{definition}


\begin{definition}
  \label{def.x.H}
  %%
  A description~$\delta$ is \emph{irreducible} (in symbols, $\delta_\#$) iff
  it is nested-irreducible and its~$\rank(\delta)$ is~$\<i,0>$, for
  some~$i\ge0$.
\end{definition}

An irreducible description~$\delta_\#=\<p,n,e>$ serves as a normal form for
transitions~$\trans$ in general.  Such description cannot be advanced
by~$\nst$, as it is nested-irreducible, and neither by~$\outpush$
nor~$\outpop$, as the second coordinate of its rank is~0, which
implies~$e=\nil$ and~$n=0$.

The next two lemmas establish that a single application of~$\out$ or~$\nst$
either preserves or decreases the rank of the input description.  All rank
comparisons assume lexicographic order, i.e., if~$\rank(\delta)=\<i,j>$
and~$\rank(\delta')=\<i',j'>$ then~$\rank(\delta)>\rank(\delta')$ iff~$i>i'$
or~$i=i$ and~$j>j'$.
%%
The proof of Lemma~\ref{lem.x.rank-out} follows directly from~\R{push}
and~\R{pop} and from Definitions~\ref{def.x.pot} and~\ref{def.x.rank}.  The
proof of Lemma~\ref{lem.x.rank-nst}, however, is by induction on the
structure of~$\nst$ derivations.

\begin{restatable}{lemma}{lemxrankout}\strut
  \label{lem.x.rank-out}
  %%
  \begin{enumerate:a}
  \item\label{lem.x.rank-out-push} If~$\delta\outpush\delta'$
    then~$\rank(\delta)=\rank(\delta')$.
  \item\label{lem.x.rank-out-pop} If~$\delta\outpop\delta'$
    then~$\rank(\delta)>\rank(\delta')$.
  \end{enumerate:a}
\end{restatable}

%\vskip-.25em

\begin{restatable}{lemma}{lemxranknst}
  \label{lem.x.rank-nst}
  %%
  If~$\delta\nst\delta'$ then~$\rank(\delta)\ge\rank(\delta')$.
\end{restatable}

The next theorem is a generalization of Lemma~\ref{lem.x.rank-nst}
for~$\nst[*]$.  Its proof follows from the lemma by induction on~$i$.

\begin{restatable}{theorem}{thmxranknstx}
  \label{thm.x.rank-nst-*}
  %%
  If~$\delta\nst[*]\delta'$ then~$\rank(\delta)\ge\rank(\delta')$.
\end{restatable}

We now state and prove the main result of this section,
Theorem~\ref{thm.x.term}, the termination theorem for~$\trans[*]$.  The idea
of the proof is that a sufficiently large sequence of~$\nst$ and~$\out$
transitions eventually decreases the rank of the current description until
an irreducible description is reached.  This irreducible description is the
final result of the reaction.
%%
\begin{restatable}[Termination]{theorem}{thmxterm}
  \label{thm.x.term}
  %%
  For any~$\delta$, there is a~$\delta'_\#$ such
  that~$\delta\trans[*]\delta'_\#$.
\end{restatable}
\begin{proof}
  By lexicographic induction on~$\rank(\delta)$.  Let~$\delta=\<p,n,e>$ and
  $\rank(\delta)=\<i,j>$.

  For the basis, suppose $\<i,j>=\<0,0>$.  Then~$\delta$ cannot be advanced
  by~$\out$, as~$j=0$ implies~$e=\nil$ and~$n=0$.  If~$\delta$ is
  nested-irreducible, the theorem is trivially true,
  as~$\delta\nst[0]\delta_\Hnst$ and~$\delta_\#$.  If~$\delta$ is not
  nested-irreducible then, by Theorem~\ref{thm.x.term-nst-*},
  $\delta\nst[*]\delta'_\Hnst$, for some~$\delta'_\Hnst$.  By
  Theorem~\ref{thm.x.rank-nst-*}, $\rank(\delta)\ge\rank(\delta')$, which
  implies~$\rank(\delta')=\<0,0>$, and so~$\delta'_\#$.

  For the inductive step, suppose~$\<i,j>>\<0,0>$.
  %%
  Then, depending on whether or not $\delta$~is nested-irreducible, there
  are two cases.
  \begin{case}
  \item\label{thm.x.term.Hnst}$\delta$~is nested-irreducible.
    %%
    If~$j=0$, by Definition~\ref{def.x.H}, $\delta_\#$, and
    so~$\delta\trans[0]\delta_\#$.  If~$j>0$, there are two subcases:
    \begin{case}
    \item\label{thm.x.term.Hnst-j>0-nonnil}$e\ne\nil$.
      %%
      Then, by~\R{push} and by Theorem~\ref{thm.x.term-nst-*}, there
      are~$\delta_1'$ and~$\delta'_\Hnst=\<p',n+1,e'>$ such that
      $\delta\outpush\delta_1'\nst[*]\delta'_\Hnst$.  Thus, by
      Lemma~\ref{lem.x.rank-out} and by Theorem~\ref{thm.x.rank-nst-*},
      \begin{align*}
        \rank(\delta)=\rank(\delta_1')=\<i,j>
        \ge\rank(\delta')=\<i',j'>\,.
      \end{align*}
      If~$e'=\nil$, then $i=i'$ and~$j=j'$, and the rest of this proof is
      similar to that of Case~\ref{thm.x.term.Hnst-j>0-nil} below.
      Otherwise, if~$e'\ne\nil$, then $i>i'$, as an~$\ceu{\EmitInt(e')}$ was
      consumed by the nested transitions.  Thus,
      $\rank(\delta)>\rank(\delta')$.  By the induction hypothesis,
      $\delta'\trans[*]\delta''_\#$, for some~$\delta''_\#$.  Therefore,
      $\delta\trans[*]\delta''_\#$.
      %%
    \item\label{thm.x.term.Hnst-j>0-nil}$e=\nil$.
      %%
      Then, as~$j>0$, $\delta\outpop\delta'$, for some~$\delta'$.  By
      Lemma~\ref{lem.x.rank-out}, $\rank(\delta)>\rank(\delta')$.  Hence, by
      the induction hypothesis, $\delta'\trans[*]\delta''_\#$, for
      some~$\delta''_\#$.  And so, $\delta\trans[*]\delta''_\#$.
    \end{case}

  \item$\delta$~is not nested-irreducible.
    %%
    Then~$e=\nil$ and, by Theorems~\ref{thm.x.term-nst-*}
    and~\ref{thm.x.rank-nst-*}, there is a~$\delta'_\Hnst$ such
    that~$\delta\nst[*]\delta'_\Hnst$
    with~$\rank(\delta)\ge\rank(\delta'_\Hnst)$.  The rest of this proof is
    similar to that of Case~\ref{thm.x.term.Hnst} above.\qedhere
  \end{case}
\end{proof}

\subsection{Memory Boundedness}

As \CEU has no mechanism for heap allocation, unbounded iteration, or
general recursion, the maximum memory usage of a given \CEU program is
determined solely by the length of its code, the number of variables it
uses, and the size of the event stack that it requires to run.  The code
length and the number of variables used are easily determined by code
inspection.  The maximum size of the event stack during a reaction of
program~$p$ to external event~$e$ corresponds to~$\pot(p,e)$, i.e., to the
maximum number of internal events that~$p$ may emit in reaction to~$e$.
If~$p$ may react to external events~$e_1$, \dots,~$e_n$ then, in the worst
case, its event stack will need to
store~$\max\{\pot(p,e_1),\dots,\pot(p,e_n)\}$ events.

% - program is finite
% - lexical scope
%     - no heap allocation
% - no code reentrancy
%     - reexecution only due to loops
%     - loop reuse nested vars
