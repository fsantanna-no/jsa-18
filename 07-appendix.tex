\section{Detailed Proofs}
\label{sec.appendix}


\lemxdetout*
\begin{proof}
  The lemma is vacuously true if~$\delta$ cannot be advanced by~$\out$
  transitions.  Suppose that is not the case, and let~$\delta$, $\delta_1$,
  and~$\delta_2$ be descriptions such that
  \begin{align*}
    \delta  &=\<\stmt,\ell,e,\theta>,\\
    \delta_1&=\<\stmt_1,\ell_1,e_1,\theta_1>,\\
    \delta_2&=\<\stmt_2,\ell_2,e_2,\theta_2>.
  \end{align*}
  Then, two cases are possible.
  \begin{case}
  \item[{[$e\ne\nil$]}] Both transitions are applications of \R{push}.
    Hence $\stmt_1=\stmt_2=\bcast(\stmt,e)$, $\ell_1=\ell_2=\ell+1$,
    and~$e_1=e_2=\nil$, and~$\theta_1=\theta_2=\theta$.
    %%
  \item[{[$e=\nil$]}] Both transitions are applications of \R{pop}.  Hence
    $\stmt_1=\stmt_2=\stmt$, $\ell_1=\ell_2=\ell-1$, and~$e_1=e_2=\nil$,
    and~$\theta_1=\theta_2=\theta$.\qedhere
  \end{case}
\end{proof}


\lemxdetnst*
\begin{proof}
  By induction on the structure of~$\nst$ derivations.  The lemma is
  vacuously true if~$\delta$ cannot be advanced by~$\nst$ transitions.
  Suppose that is not the case and let~$\delta$, $\delta_1$, and~$\delta_2$
  be descriptions such that
  \begin{align*}
    \delta  &=\<\stmt,\ell,e,\theta>,\\
    \delta_1&=\<\stmt_1,\ell_1,e_1,\theta_1>,\\
    \delta_2&=\<\stmt_2,\ell_2,e_2,\theta_2>.
  \end{align*}
  Then, by the hypothesis of the lemma, there are derivations~$\pi_1$
  and~$\pi_2$ such that
  \[
    \pi_1=\null
    \AxiomC{\strut$\vdots$}
    \UnaryInfC{\strut$\delta\to\delta_1$}
    \raisebox{5pt}{\DisplayProof}
    \quad\text{and}\quad
    \pi_2=\null
    \AxiomC{\strut$\vdots$}
    \UnaryInfC{$\delta\to\delta_2$}
    \raisebox{5pt}{\DisplayProof}
  \]
  That is, the conclusion of derivation~$\pi_1$ is~$\delta\nst\delta_1$ and
  the conclusion of derivation~$\pi_2$ is~$\delta\nst\delta_2$.

  By definition of~$\nst$, we have that~$e=\nil$ and $\ell_1=\ell_2=\ell$.
  It remains to be shown that~$\stmt_1=\stmt_2$, $e_1=e_2$,
  and~$\theta_1=\theta_2$.

  Depending on the structure of program~$\stmt$, the following cases are
  possible.  (Note that~$\stmt$ cannot be an~$\ceu{\AwaitExt}$,
  $\ceu{\AwaitInt}$, $\ceu{\Break}$, $\ceu{\Every}$, $\ceu{\Fin}$,
  or~$\ceu{\Nop}$ statement as there is no~$\nst$ rule to advance such
  programs.)

  \begin{case}
  \item[{[$\stmt=\ceu{\Var{v\,\stmt'}}$]}] Then~$\pi_1$ and~$\pi_2$ are
    instances of \R{var-expd}, i.e., their conclusions are obtained by an
    application of this rule.  Hence,
    \[
      \stmt_1=\stmt_2=\ceu{\AtVar{v\,\bot\,\stmt'}},
    \]
    $e_1=e_2=\nil$, and~$\theta_1=\theta_2=\theta$.
    %%
  \item[{[$\stmt=\ceu{v\coloneqq\expr}$]}] Then~$\pi_1$ and~$\pi_2$ are
    instances of \R{assign}.  Hence, $\stmt_1=\stmt_2=\ceu{\Nop}$,
    $e_1=e_2=\nil$,
    and~$\theta_1=\theta_2=\updt(\theta,v,\eval(\theta,\expr))$.
    %%
  \item[{[$\stmt=\ceu{\EmitInt(e')}$]}] Then~$\pi_1$ and~$\pi_2$ are
    instances of \R{emit-int}.  Hence,
    $\stmt_1=\stmt_2=\ceu{\RunAt(\ell)}$, $e_1=e_2=\nil$,
    and~$\theta_1=\theta_2=\theta$.
    %%
  \item[{[$\stmt=\ceu{\IfElse{\expr}{\stmt'}{\stmt''}}$]}]
    %%
    If $\eval(\theta,\expr)\ne0$, then~$\pi_1$ and~$\pi_2$ are instances of
    \R{if-true}.  Hence, $\stmt_1=\stmt_2=\stmt'$, $e_1=e_2=\nil$, and
    $\theta_1=\theta_2=\theta$.  The case for~$\eval(\theta,\expr)=0$ is
    similar but uses \R{if-false}.
    %%
  \item[{[$\stmt=\ceu{\stmt';\stmt''}$]}]
    There are three subcases.
    \begin{case}
    \item[{[$\stmt'=\ceu{\Nop}$]}] Then~$\pi_1$ and~$\pi_2$ are instances of
      \R{seq-nop}.  Hence, $\stmt_1=\stmt_2=\stmt''$, $e_1=e_2=\nil$, and
      $\theta_1=\theta_2=\theta$.
    \item[{[$\stmt'=\ceu{\Break}$]}] Then~$\pi_1$ and~$\pi_2$ are instances
      of \R{seq-brk}. Hence, $\stmt_1=\stmt_2=\ceu{\Break}$, $e_1=e_2=\nil$,
      and $\theta_1=\theta_2=\theta$.
    \item[{[$\stmt'\ne\ceu{\Nop,\Break}$]}] Then~$\pi_1$ and~$\pi_2$ are
      instances of \R{seq-adv}.  Hence, there are
      derivations~$\pi_1'\prec\pi_1$ and~$\pi_2'\prec\pi_2$ (i.e.,
      structurally smaller than~$\pi_1$ and~$\pi_2$) such that
      \begin{align*}
        \pi_1'&=\null
        \AxiomC{\strut$\vdots$}
        \UnaryInfC{$\<\stmt',\ell,\nil,\theta>
          \nst\<\stmt_1',\ell,e_1',\theta_1'>$}
        \raisebox{5pt}{\DisplayProof}\\
        \shortintertext{and}
        \pi_2'&=\null
                \AxiomC{\strut$\vdots$}
        \UnaryInfC{$\<\stmt',\ell,\nil,\theta>
          \nst\<\stmt_2',\ell,e_2',\theta_2'>$}
        \raisebox{5pt}{\DisplayProof}
      \end{align*}
      By the induction hypothesis, $\stmt_1'=\stmt_2'$, $e_1'=e_2'$
      and~$\theta_1'=\theta_2'$.  Therefore,
      \[
        \stmt_1=\ceu{\stmt_1';\stmt''}=\ceu{\stmt_2';\stmt''}=\stmt_2,
      \]
      $e_1=e_1'=e_2'=e_2$, and $\theta_1=\theta_1'=\theta_2'=\theta_2$.
    \end{case}
    %%
  \item[{[$\stmt=\ceu{\Loop{\stmt'}}$]}] Similar to the case for
    $\stmt=\ceu{\Var{v\,\stmt'}}$ but with rule \R{loop-expd}.
    %%
  \item[{[$\stmt=\ceu{\stmt'\ParAnd\stmt''}$]}] Ditto (rule
    \R{par/and-expd}).
    %%
  \item[{[$\stmt=\ceu{\stmt'\ParOr\stmt''}$]}] Ditto (rule \R{par/or-expd}).
    %%
  \item[{[$\stmt=\ceu{\RunAt(\ell)}$]}] Then~$\pi_1$ and~$\pi_2$ are
    instances of \R{can-run}.  Hence, $\stmt_1=\stmt_2=\ceu{\Nop}$,
    $e_1=e_2=\nil$, and $\theta_1=\theta_2=\theta$.
    %%
  \item[{[$\stmt=\ceu{\AtVar{v\,n,\stmt'}}$]}] Similar to the case for
    $\stmt=\ceu{\stmt';\stmt''}$ but with rules \R{var-nop}, \R{var-brk},
    and \R{var-adv}.
    %%
  \item[{[$\stmt=\ceu{\stmt'\AtLoop\stmt''}$]}] Ditto (rules \R{loop-nop},
    \R{loop-brk}, and \R{loop-adv}).
    %%
  \item[{[$\stmt=\ceu{\stmt'\AtParAnd\stmt''}$]}] There are two subcases.
    \begin{case}
    \item[{[$\lnot\isblk(\stmt')$]}] Then $\pi_1$ and~$\pi_2$ are instances
      of \R{par/and-nop1}, \R{par/and-brk1}, or \R{par/and-adv1}.  The proof
      of this case is similar to that of the cases where rules \R{seq-nop},
      \R{seq-brk}, and \R{seq-adv} apply.
    \item[{[$\isblk(\stmt')$]}] Ditto but with $\pi_1$ and $\pi_2$ as
      instances of \R{par/and-nop2}, \R{par/and-brk2}, and \R{par/and-adv2}.
    \end{case}
  \item[{[$\stmt=\ceu{\stmt'\AtParOr\stmt''}$]}] Ditto (rules
    \R{par/or-nop1}, \R{par/or-brk1}, \R{par/or-adv1}, \R{par/or-nop2},
    \R{par/or-brk2}, and \R{par/or-adv2}).\qedhere
  \end{case}
\end{proof}


\propxirrnsti*
\begin{proof}
  By contradiction on the hypothesis that there is such~$k$.
  %%
  Let~$\delta\nst[i]\delta'_\Hnst$, for some~$i\ge0$.
  %%
  There are two cases.
  \begin{case}
  \item\label{prop.x.irr-nst-i-case1}
    %%
    Suppose there are~$k>i$ and~$\delta''_\Hnst$ such
    that~$\delta\nst[k]\delta''$.
    %%
    Then, by definition of~$\nst[k]$,
    \begin{equation}
      \label{prop.x.irr-nst-i-eq1}
      %%
      \delta\nst[i]\delta'\nst[i+1]\delta_1'\nst[i+2]\cdots\nst[k]\delta''.
    \end{equation}
    Since~$\delta'=\<\stmt',\ell,e',\theta'>$ is nested-irreducible,
    $e'=\nil$ or~$\stmt'=\ceu{\Nop}$ or~$\stmt'=\ceu{\Break}$
    or~$\isblk(\stmt',\ell)$.  In any of these cases, by the definition
    of~$\nst$, there is no~$\delta_1'$ such that~$\delta'\nst[1]\delta_1'$,
    which contradicts~\eqref{prop.x.irr-nst-i-eq1}.  Therefore, no such~$k$
    can exist.
    %%
  \item Suppose there are~$k<i$ and~$\delta''_\Hnst$ such
    that~$\delta\nst[k]\delta''$.  Then, since~$i>k$, by
    Case~\ref{prop.x.irr-nst-i-case1}, $\delta'$~cannot exist, which is
    absurd.  Therefore, the assumption that there is such~$k$ is
    false.\qedhere
  \end{case}
\end{proof}


\lemxpropsnstia*
\begin{proof}
  By induction on~$i$.  We will only prove case for the $\ceu{\AtParOr}$
  composition.  The proof of the other cases is similar.

  So, considering the case for the $\ceu{\AtParOr}$, the lemma is trivially
  true for~$i=0$, as~$\stmt_1=\stmt_1'$, and follows directly from rule
  \R{par/or-adv1} for~$i=1$.  Suppose
  \begin{align*}
    \<\stmt_1,\ell,e,\theta>
    &\nst[1]\<\stmt_1'',\ell,e'',\theta''>\\
    &\nst[i-1]\<\stmt_1',\ell,e',\theta'>\,,
  \end{align*}
  for some~$i>1$.  Then~$\<\stmt_1'',\ell,e'',\theta''>$ is not
  nested-irreducible, and by rule \R{par/or-adv1},
  \[
    \<\ceu{{\stmt_1}\AtOr{\stmt_2}},\ell,e,\theta>
    \nst[1]\<\ceu{{\stmt_1}''\AtOr{\stmt_2}},\ell,e'',\theta''>.
  \]
  Since
  \[
    \<\stmt_1'',\ell,e'',\theta''>
    \nst[i-1]\<\stmt_1',\ell,e',\theta'>\,,
  \]
  by the induction hypothesis,
  \[
    \<\ceu{{\stmt_1''}\AtOr{\stmt_2}},\ell,e'',\theta''>
    \nst[i-1]\<\ceu{{\stmt_1'}\AtOr{\stmt_2}},\ell,e',\theta'>.
  \]
  Therefore,
  \[
    \<\ceu{{\stmt_1}\AtOr{\stmt_2}},\ell,e,\theta>
    \nst[i]\<\ceu{{\stmt_1'}\AtOr{\stmt_2}},\ell,e',\theta'>.\qedhere
  \]
\end{proof}


\lemxpropsnstib*
\begin{proof}
  Similar to the proof of Lemma~\ref{lem.x.props-nst-i-a}.
\end{proof}


\propxsynrestloop*
\gl{TODO}


\propxsynrestfin*
\gl{TODO}


\thmxtermnstx*
\begin{proof}
  By induction on the structure of programs.
  Let~$\delta=\<\stmt,\ell,\nil,\theta>$.  The theorem is trivially true
  if~$\delta$ is nested-irreducible, as by definition~$\delta\nst[0]\delta$.

  Suppose~$\delta$ is not nested irreducible.  Then, depending on the
  structure of~$\stmt$, the following cases are possible (these are the only
  values for $\stmt$ that do not make $\delta$ nested-irreducible).  In each
  of the cases, we show that the required description~$\delta'_\Hnst$
  exists.

  \begin{case}
  \item[{[$stmt=\ceu{\Var{v\,\stmt'}}$]}]
    %%
  \item[{[$stmt=\ceu{v\coloneqq\expr}$]}] By \R{assign},
    \[
      \<\ceu{v\coloneqq\expr},\ell,\nil,\theta>
      \nst[1]\<\ceu{\Nop},\ell,\nil,\updt(\theta,v,\eval(\theta,\expr))>_\Hnst.
    \]
    %%
  \item[{[$stmt=\ceu{\EmitInt(e)}$]}] By \R{emit-int},
    \[
      \<\ceu{\EmitInt(e)},\ell,\nil,\theta>
      \nst[1]\<\ceu{\RunAt(\ell)},\ell,e,\theta>_\Hnst.
    \]
    %%
  \item[{[$stmt=\ceu{\IfElse{\expr}{\stmt'}{\stmt''}}$]}]
    If~$\eval(\theta,\expr)\ne0$ then, by \R{if-true} and by the induction
    hypothesis, there is a~$\delta'$ such that
    \begin{align*}
      \<\ceu{\IfElse{\expr}{\stmt'}{\stmt''}},\ell,\nil,\theta>
      &\nst[1]\<stmt',\ell,\nil,\theta>\\
      &\nst[*]\delta'_\Hnst.
    \end{align*}
    The case for~$\eval(\theta,\expr)=0$ is similar but uses \R{if-false}.
    %%
  \item[{[$stmt=\ceu{\stmt';\stmt''}$]}]
    %%
    There are three subcases.
    \begin{case}
    \item[{[$stmt'=\ceu{\Nop}$]}] By \R{seq-nop} and by the induction
      hypothesis, there is a~$\delta'$ such that
      \begin{align*}
        \<\ceu{\Nop;\stmt''},\ell,\nil,\theta>
        &\nst[1]\<\ceu{\stmt''},\ell,\nil\,theta>\\
        &\nst[*]\delta'_\Hnst.
      \end{align*}
    \item[{[$stmt'=\ceu{\Break}$]}] By \R{seq-brk},
      \[
        \<\ceu{\Break;\stmt''},\ell,\nil,\theta>
        \nst[1]\<\ceu{\Break},\ell,\nil,\theta>_\Hnst.
      \]
    \item[{[$stmt'\ne\ceu{\Nop,\Break}$]}] By induction hypothesis,
      there are~$\stmt_1'$, $e$, and~$\theta'$ such that
      \[
        \<\stmt',\ell,\nil,\theta>\nst[*]\<\stmt_1',\ell,e,\theta'>_\Hnst.
      \]
      Then, by Lemma~\ref{lem.x.props-nst-i-a},
      \[
        \<\ceu{\stmt';\stmt''},\ell,\nil,\theta>
        \nst[*]\<\ceu{\stmt_1';\stmt''},\ell,e,\theta>.
      \]
      It remains to be shown that description
      \[
        \delta'=\<\ceu{\stmt_1';\stmt''},\ell,\nil,\theta>
      \]
      is nested-irreducible or leads to a nested irreducible description.
      This follows from the fact that the simpler description
      $\<\stmt_1',\ell,e,\theta'>$ is nested-irreducible:
      \begin{enumerate*}[label=(\roman*)]
      \item if~$e\ne\nil$ then~$\delta'_\Hnst$;
      \item if~$\stmt_1'=\ceu{\Nop}$ or~$\stmt_1'=\ceu{\Break}$ then the
        rest of the proof is similar to the cases
        for~$\stmt=\ceu{\Nop;\stmt''}$ and~$\stmt=\ceu{\Break;\stmt''}$; and
      \item if~$\isblk(\stmt_1',\ell)$ then, by definition,
        $\isblk(\stmt_1';\stmt_2,\ell)$, and so~$\delta'_\Hnst$.
      \end{enumerate*}
    \end{case}
    %%
  \item[{[$stmt=\ceu{\Loop{\stmt'}}$]}]
    By Proposition~\ref{prop.x.syn-rest.loop},
    \[
      \<\ceu{\Loop{\stmt'}},\ell,\nil,\theta>
      \nst[*]\<\stmt'',\ell,e,\theta'>
    \]
    %%
    where either $\stmt''=\ceu{\Break\AtLoop{\stmt'}}$
    or~$\isblk(\stmt'',\ell)$.  In the first case, by \R{loop-brk},
    \begin{align*}
      \<\ceu{\Loop{\stmt'}},\ell,\nil,\theta>
      &\nst[*]\<\ceu{\Break\AtLoop{\stmt'}},\ell,e,\theta'>\\
      &\nst[1]\<\ceu{\Nop},\ell,e,\theta'>_\Hnst.
    \end{align*}
    In the second case, by definition, $\<\stmt'',\ell,e,\theta'>_\Hnst$.
    %% 
  \item[{[$stmt=\ceu{\stmt'\ParAnd\stmt''}$]}]  By \R{par/and-expd},
    \begin{align*}
      &\<\ceu{\stmt'\ParAnd\stmt''},\ell,\nil,\theta>\\[-\jot]
      &\nst[1]\<\ceu{\stmt1\AtParAnd(\RunAt(\ell);\stmt'')},\ell,\nil,\theta>.
    \end{align*}
    From this point on, this case becomes that
    for~$\stmt=\ceu{\stmt'\AtParAnd\stmt''}$.
    %% 
  \item[{[$stmt=\ceu{\stmt'\ParOr\stmt''}$]}] Similar to the case
    for~$\stmt=\ceu{\stmt'\ParAnd\stmt''}$.
    %%
  \item[{[$stmt=\ceu{\RunAt(\ell)}$]}] By \R{run-at},
    \[
      \<\ceu{\RunAt(\ell)},\ell,\nil,\theta>
      \nst[1]\<\ceu{\Nop},\ell,\nil,\theta>_\Hnst.
    \]
    %%
  \item[{[$stmt=\ceu{\AtVar{v\,n\,\stmt'}}$]}] Trivial if~$\stmt'$
    is~$\ceu{\Nop}$ or~$\ceu{\Break}$.  Suppose that is not the case.  Then,
    by induction hypothesis,
    \[
      \<\stmt',\ell,\nil,(v,n):\theta>
      \nst[*]\<\stmt'',\ell,e,(v,n'):\theta'>_\Hnst.
    \]
    By Lemma~\ref{lem.x.props-nst-i-a},
    \[
      \<\ceu{\AtVar{v\,n\,\stmt'}},\ell,\nil,\theta>
      \nst[*]\<\ceu{\AtVar{v\,n',\stmt''}},\ell,e,\theta'>.
    \]
    Again,...
    %%
  \item[{[$stmt=\ceu{\stmt'\AtLoop{\stmt''}}$]}]
    There are three subcases.
    \begin{case}
    \item[{[$stmt'=\ceu{\Nop}$]}] By \R{loop-nop},
      \[
        \<\ceu{\Nop\AtLoop{\stmt''}},\ell,\nil,\theta>
        \nst[1]\<\ceu{\Loop{\stmt''}},\ell,\nil,\theta>.
      \]
      The rest of the proof is similar to the case
      for~$\stmt=\ceu{\Loop{\stmt'}}$.
    \item[{[$stmt'=\ceu{\Break}$]}] By \R{loop-brk},
      \[
        \<\ceu{\Break\AtLoop{\stmt''}},\ell,\nil,\theta>
        \nst[1]\<\ceu{\Nop},\ell,\nil,\theta>_\Hnst.
      \]
    \item[{[$stmt'\ne\ceu{\Nop,\Break}$]}] Similar to the case
      for~$\stmt={\stmt';\stmt''}$ with~$\stmt'\ne\ceu{\Nop,\Break}$.
    \end{case}
    %%
  \item[{[$stmt=\ceu{\stmt'\AtParAnd\stmt''}$]}] Suppose that $\stmt'$ is
    not blocked, i.e., $\lnot\isblk(\stmt',\ell)$.  (The proof for a blocked
    $\stmt'$ is similar.)  There are three subcases.
    \begin{case}
    \item[{[$stmt'=\ceu{\Nop}$]}] By \R{par/and-nop1} and by the induction
      hypothesis, there is a~$\delta'$ such that
      \[
        \<\ceu{\Nop\AtParAnd\stmt''},\ell,\nil,\theta>
        \nst[1]\<\ceu{\stmt''},\ell,\nil\theta>
        \nst[*]\delta'_\Hnst.
      \]
    \item[{[$stmt'=\ceu{\Break}$]}] By \R{par/and-brk1},
      \begin{align*}
        &\<\ceu{\Break\AtParAnd\stmt''},\ell,\nil,\theta>\\[-\jot]
        &\qquad\nst[1]\<\ceu{\clear(\stmt'');\Break},\ell,\nil,\theta>.
      \end{align*}
      By Proposition~\ref{prop.x.syn-rest.fin},
      \[
        \<\ceu{\clear(\stmt'')},\ell,\nil,\theta>.
        \nst[*]\<\stmt''',\ell,e,\theta'>_\Hnst.
      \]
      Hence, by Lemma~\ref{lem.x.props-nst-i-a},
      \begin{align*}
        &\<\ceu{\clear(\stmt'');\Break},\ell,\nil,\theta>\\[-\jot]
        &\qquad\nst[*]\<\ceu{\stmt''';\Break},\ell,e,\theta'>_\Hnst.
      \end{align*}
    \item[{[$stmt'\ne\ceu{\Nop,\Break}$]}] By the induction hypothesis,
      \[
        \<\stmt',\ell,\nil,\theta>\nst[*]\<\stmt_1',\ell,e,\theta'>_\Hnst.
      \]
      By Lemma~\ref{lem.x.props-nst-i-a},
      \begin{align*}
        &\<\ceu{\stmt'\AtParAnd\stmt''},\ell,\nil,\theta>\\[-\jot]
        &\qquad\nst[*]\<\ceu{\stmt_1'\AtParAnd\stmt''},\ell,e,\theta>.
      \end{align*}
      It remains to be shown that description
      \[
        \delta'=\<\ceu{\stmt_1'\AtParAnd\stmt''},\ell,e,\theta>
      \]
      is nested-irreducible or leads to a nested-irreducible description.
      This proof is similar to that used in the case
      for~$\ceu{\stmt=\stmt';\stmt''}$ with~$\stmt'\ne\ceu{\Nop,\Break}$.
      That is, if $e\ne\nil$ then~$\delta_\Hnst$;
      if~$\stmt_1'=\ceu{\Nop,\Break}$ then this case reduces to those
      for~$\stmt=\ceu{\Nop\AtParAnd\stmt''}$
      and~$\stmt=\ceu{\Break\AtParAnd\stmt''}$; and if~$\isblk(stmt_1'')$
      then this case reduces to that for~$\ceu{\stmt'\AtParAnd\stmt''}$
      with~$\isblk(stmt',\ell)$ evaluating to true.
    \end{case}
    %%
  \item[{[$stmt=\ceu{\stmt'\AtParOr\stmt''}$]}] Similar to the case for
    $\stmt=\ceu{\stmt'\AtParAnd\stmt''}$.\qedhere
  \end{case}
\end{proof}


%%% Local Variables:
%%% mode: latex
%%% TeX-master: "index.tex"
%%% End:
