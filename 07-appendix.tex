\section{Detailed Proofs}
\label{sec.appendix}


\lemdetout*
\begin{proof}
  The lemma is vacuously true if $\delta$ cannot be advanced by $\out$
  transitions.  Suppose that is not the case, and let $\delta$, $\delta_1$,
  and $\delta_2$ be descriptions such that
  \begin{align*}
    \delta  &=\<\stmt,\ell,e,\theta>,\\
    \delta_1&=\<\stmt_1,\ell_1,e_1,\theta_1>,\\
    \delta_2&=\<\stmt_2,\ell_2,e_2,\theta_2>.
  \end{align*}
  Then, two cases are possible.
  \begin{case}
  \item[{[$e\ne\nil$]}] Both transitions are applications of \R{push}.
    Hence, $\stmt_1=\stmt_2=\bcast(\stmt,e)$, $\ell_1=\ell_2=\ell+1$,
    $e_1=e_2=\nil$, and $\theta_1=\theta_2=\theta$.
    %%
  \item[{[$e=\nil$]}] Both transitions are applications of \R{pop}.  Hence,
    $\stmt_1=\stmt_2=\stmt$, $\ell_1=\ell_2=\ell-1$, $e_1=e_2=\nil$,
    and $\theta_1=\theta_2=\theta$.\qedhere
  \end{case}
\end{proof}


\lemdetnst*
\begin{proof}
  By induction on the structure of $\nst$ derivations.  The lemma is
  vacuously true if $\delta$ cannot be advanced by $\nst$ transitions.
  Suppose that is not the case and let $\delta$, $\delta_1$, and $\delta_2$
  be descriptions such that
  \begin{align*}
    \delta  &=\<\stmt,\ell,e,\theta>,\\
    \delta_1&=\<\stmt_1,\ell,e_1,\theta_1>,\\
    \delta_2&=\<\stmt_2,\ell,e_2,\theta_2>.
  \end{align*}
  Then, by the hypothesis of the lemma, there are derivations $\pi_1$
  and $\pi_2$ such that
  \[
    \AxiomC{\strut$\vdots$}
    \LeftLabel{\strut$\pi_1=\null$}
    \UnaryInfC{\strut$\delta\to\delta_1$}
    \DisplayProof
    \quad\text{and}\quad
    \AxiomC{\strut$\vdots$}
    \LeftLabel{\strut$\pi_2=\null$}
    \UnaryInfC{\strut$\delta\to\delta_2$}
    \DisplayProof
  \]
  That is, the conclusion of derivation $\pi_1$ is $\delta\nst\delta_1$ and
  the conclusion of derivation $\pi_2$ is $\delta\nst\delta_2$.

  By definition of $\nst$, $e=\nil$.  It remains to be shown that
  $\stmt_1=\stmt_2$, $e_1=e_2$, and $\theta_1=\theta_2$.  Depending on the
  structure of program $\stmt$, the following cases are possible.  (Note
  that these are the only values for $\stmt$ that permit an application of
  $\nst$.)

  \begin{case}
  \item[{[$\ceu{\Var{v\,\stmt'}}$]}] Then $\pi_1$ and $\pi_2$ are instances
    of \R{var-expd}, i.e., their conclusions are obtained by an application
    of this rule.  Hence, $\stmt_1=\stmt_2=\ceu{\AtVar{v\,\bot\,\stmt'}}$,
    $e_1=e_2=\nil$, and $\theta_1=\theta_2=\theta$.
    %%
  \item[{[$\ceu{v\coloneqq\expr}$]}] Then $\pi_1$ and $\pi_2$ are instances
    of \R{assign}.  Hence, $\stmt_1=\stmt_2=\ceu{\Nop}$, $e_1=e_2=\nil$,
    and $\theta_1=\theta_2=\updt(\theta,v,\eval(\theta,\expr))$.
    %%
  \item[{[$\ceu{\EmitInt(e')}$]}] Then $\pi_1$ and $\pi_2$ are instances of
    \R{emit-int}.  Hence, $\stmt_1=\stmt_2=\ceu{\RunAt(\ell)}$,
    $e_1=e_2=\nil$, and $\theta_1=\theta_2=\theta$.
    %%
  \item[{[$\ceu{\IfElse{\expr}{\stmt'}{\stmt''}}$]}]
    %%
    If $\eval(\theta,\expr)\ne0$, then $\pi_1$ and $\pi_2$ are instances of
    \R{if-true}.  Hence, $\stmt_1=\stmt_2=\stmt'$, $e_1=e_2=\nil$, and
    $\theta_1=\theta_2=\theta$.  The case for $\eval(\theta,\expr)=0$ is
    similar.
    %%
  \item[{[$\ceu{\stmt';\stmt''}$]}] There are three subcases.
    \begin{case}
    \item[{[$\stmt'=\ceu{\Nop}$]}] Then $\pi_1$ and $\pi_2$ are instances of
      \R{seq-nop}.  Hence, $\stmt_1=\stmt_2=\stmt''$, $e_1=e_2=\nil$, and
      $\theta_1=\theta_2=\theta$.
    \item[{[$\stmt'=\ceu{\Break}$]}] Then $\pi_1$ and $\pi_2$ are instances
      of \R{seq-brk}. Hence, $\stmt_1=\stmt_2=\ceu{\Break}$, $e_1=e_2=\nil$,
      and $\theta_1=\theta_2=\theta$.
    \item[{[$\stmt'\ne\ceu{\Nop,\Break}$]}] Then $\pi_1$ and $\pi_2$ are
      instances of \R{seq-adv}.  Hence, there are
      derivations $\pi_1'\prec\pi_1$ and $\pi_2'\prec\pi_2$ (i.e.,
      structurally smaller than $\pi_1$ and $\pi_2$) such that
      \[
        \AxiomC{\strut$\vdots$}
        \LeftLabel{\strut$\pi_1'=\null$}
        \UnaryInfC{\strut$\<\stmt',\nil,\theta>
          \nst\<\stmt_1',e_1',\theta_1'>$}
        \DisplayProof
      \]
      and
      \[
        \AxiomC{\strut$\vdots$}
        \LeftLabel{\strut$\pi_2'=\null$}
        \UnaryInfC{\strut$\<\stmt',\nil,\theta>
          \nst\<\stmt_2',e_2',\theta_2'>$}
        \DisplayProof
      \]
      By the induction hypothesis, $\stmt_1'=\stmt_2'$, $e_1'=e_2'$
      and $\theta_1'=\theta_2'$.  Therefore,
      \[
        \stmt_1=\ceu{\stmt_1';\stmt''}=\ceu{\stmt_2';\stmt''}=\stmt_2,
      \]
      $e_1=e_1'=e_2'=e_2$, and $\theta_1=\theta_1'=\theta_2'=\theta_2$.
    \end{case}
    %%
  \item[{[$\ceu{\Loop{\stmt'}}$]}] Similar to the case for
    variable declarations ($\ceu{\Var}$).
    %%
  \item[{[$\ceu{\stmt'\ParAnd\stmt''}$]}] Ditto.
    %%
  \item[{[$\ceu{\stmt'\ParOr\stmt''}$]}] Ditto.
    %%
  \item[{[$\ceu{\RunAt(\ell)}$]}] Then $\pi_1$ and $\pi_2$ are
    instances of \R{run-at}.  Hence, $\stmt_1=\stmt_2=\ceu{\Nop}$,
    $e_1=e_2=\nil$, and $\theta_1=\theta_2=\theta$.
    %%
  \item[{[$\ceu{\AtVar{v\,n\,stmt'}}$]}] Similar to the case for sequences.
    %%
  \item[{[$\ceu{\stmt'\AtLoop\stmt''}$]}] Ditto.
    %%
  \item[{[$\ceu{\stmt'\AtParAnd\stmt''}$]}] There are two subcases.
    \begin{case}
    \item[{[$\lnot\isblk(\stmt',\ell)$]}] Then $\pi_1$ and $\pi_2$ are
      instances of \R{par/and-nop1}, \R{par/and-brk1}, or \R{par/and-adv1}.
      The rest of the proof of is similar to that of the case for sequences.
    \item[{[$\isblk(\stmt',\ell)$]}] Similar but with $\pi_1$ and $\pi_2$ as
      instances of \R{par/and-nop2}, \R{par/and-brk2}, and \R{par/and-adv2}.
    \end{case}
  \item[{[$\ceu{\stmt'\AtParOr\stmt''}$]}] Similar to the case for
    $\ceu{\AtParAnd}$.\qedhere
  \end{case}
\end{proof}


\propirrnsti*
\begin{proof}
  By contradiction on the hypothesis that there is such $k$.
  %%
  Let $\delta\nst[i]\delta'_\Hnst$, for some $i\ge0$.
  %%
  There are two cases.
  \begin{case}
  \item\label{prop.irr-nst-i-case1}
    %%
    Suppose there are $k>i$ and $\delta''_\Hnst$ such
    that $\delta\nst[k]\delta''$.
    %%
    Then, by definition of $\nst[k]$,
    \begin{equation}
      \label{prop.irr-nst-i-eq1}
      %%
      \delta\nst[i]\delta'\nst[i+1]\delta_1'\nst[i+2]\cdots\nst[k]\delta''.
    \end{equation}
    Since $\delta'=\<\stmt',\ell,e',\theta'>$ is nested-irreducible,
    $e'=\nil$ or $\stmt'=\ceu{\Nop}$ or $\stmt'=\ceu{\Break}$
    or $\isblk(\stmt',\ell)$.  In any of these cases, by the definition
    of $\nst$, there is no $\delta_1'$ such that $\delta'\nst[1]\delta_1'$,
    which contradicts \eqref{prop.irr-nst-i-eq1}.  Therefore, no such $k$
    can exist.
    %%
  \item Suppose there are $k<i$ and $\delta''_\Hnst$ such
    that $\delta\nst[k]\delta''$.  Then, since $i>k$, by
    Case \ref{prop.irr-nst-i-case1}, $\delta'$ cannot exist, which is
    absurd.  Therefore, the assumption that there is such $k$ is
    false.\qedhere
  \end{case}
\end{proof}


\lempropsnsti*
\begin{proof}
  By induction on $i$.  We will only prove the case for the $\ceu{\AtParOr}$
  composition in item~\eqref{lem.props-nst-i.1}.  The proof of the other
  cases is similar.

  So, considering the case for the $\ceu{\AtParOr}$ in
  item~\eqref{lem.props-nst-i.1}, the lemma is trivially true for $i=0$,
  as $\stmt_1=\stmt_1'$, and follows directly from \R{par/or-adv1} for
  $i=1$.  Suppose
  \begin{align*}
    \<\stmt_1,\nil,\theta>
    &\nst[1]\<\stmt_1'',e'',\theta''>\\
    &\nst[i-1]\<\stmt_1',e,\theta'>\,,
  \end{align*}
  for some $i>1$.  Then $\<\stmt_1'',e,\theta''>$ is not nested-irreducible,
  and so, by \R{par/or-adv1},
  \[
    \<\ceu{\stmt_1\AtParOr\stmt_2},\nil,\theta>
    \nst[1]\<\ceu{\stmt_1''\AtParOr\stmt_2},e'',\theta''>.
  \]
  Since
  \[
    \<\stmt_1'',e'',\theta''>
    \nst[i-1]\<\stmt_1',e,\theta'>\,,
  \]
  by the induction hypothesis,
  \[
    \<\ceu{\stmt_1''\AtParOr\stmt_2},e'',\theta''>
    \nst[i-1]\<\ceu{\stmt_1'\AtParOr\stmt_2},e,\theta'>.
  \]
  Therefore,
  \[
    \<\ceu{\stmt_1\AtParOr\stmt_2},\nil,\theta>
    \nst[i]\<\ceu{\stmt_1'\AtParOr\stmt_2},e,\theta'>.\qedhere
  \]
\end{proof}


% \propsynrestloop*
% \propsynrestfin*


\thmtermnstx*
\begin{proof}
  By induction on the structure of statements.  The theorem is trivially
  true if $\delta$ is nested-irreducible, as by definition
  $\delta\nst[0]\delta$.

  Suppose $\delta=\<\stmt,\ell,\nil,\theta>$ is not nested-irreducible.
  Then, depending on the structure of $\stmt$, the following cases are
  possible (these are the only values for $\stmt$ that do not make $\delta$
  nested-irreducible).  In each of the cases, we show that the required
  description $\delta'_\Hnst$ exists.

  \begin{case}
  \item[{[$\ceu{\Var{v\,\stmt'}}$]}] By \R{var-expd},
    \[
      \<\ceu{\Var{v\,\stmt'}},\nil,\theta>
      \nst[1]\<\ceu{\Var{v\,\bot\,\stmt'}},\nil,\theta>.
    \]
    The rest of the proof is similar to that of the case for~$\ceu{\AtVar}$.
    %%
  \item[{[$\ceu{v\coloneqq\expr}$]}] By \R{assign},
    \begin{align*}
      &\<\ceu{v\coloneqq\expr},\ell,\nil,\theta>\\
      &\qquad\nst[1]\<\ceu{\Nop},\ell,\nil,
        \updt(\theta,v,\eval(\theta,\expr))>_\Hnst.
    \end{align*}
    %% 
  \item[{[$\ceu{\EmitInt(e)}$]}] By \R{emit-int},
    \[
      \<\ceu{\EmitInt(e)},\ell,\nil,\theta>
      \nst[1]\<\ceu{\RunAt(\ell)},\ell,e,\theta>_\Hnst.
    \]
    %%
  \item[{[$\ceu{\IfElse{\expr}{\stmt'}{\stmt''}}$]}] If
    $\eval(\theta,\expr)\ne0$ then, by \R{if-true} and by the induction
    hypothesis, for some~$\delta'$,
    \begin{align*}
      \<\ceu{\IfElse{\expr}{\stmt'}{\stmt''}},\nil,\theta>
      &\nst[1]\<stmt',\nil,\theta>\\
      &\nst[*]\delta'_\Hnst.
    \end{align*}
    The case for $\eval(\theta,\expr)=0$ is similar.
    %%
  \item[{[$\ceu{\stmt';\stmt''}$]}]
    %%
    There are three subcases.
    \begin{case}
    \item[{[$stmt'=\ceu{\Nop}$]}] By \R{seq-nop} and by the induction
      hypothesis, for some~$\delta'$,
      \begin{align*}
        \<\ceu{\Nop;\stmt''},\nil,\theta>
        &\nst[1]\<\ceu{\stmt''},\nil\,theta>\\
        &\nst[*]\delta'_\Hnst.
      \end{align*}
    \item[{[$stmt'=\ceu{\Break}$]}] By \R{seq-brk},
      \[
        \<\ceu{\Break;\stmt''},\nil,\theta>
        \nst[1]\<\ceu{\Break},\nil,\theta>_\Hnst.
      \]
    \item[{[$stmt'\ne\ceu{\Nop,\Break}$]}] By induction hypothesis,
      for some~$\stmt_1'$,
      \[
        \<\stmt',\nil,\theta>\nst[*]\<\stmt_1',e,\theta'>_\Hnst.
      \]
      Then, by Lemma \ref{lem.props-nst-i},
      \[
        \<\ceu{\stmt';\stmt''},\nil,\theta>
        \nst[*]\<\ceu{\stmt_1';\stmt''},e,\theta'>=\delta'.
      \]
      It remains to be shown that $\delta'$ is nested-irreducible or leads
      to a nested-irreducible description.  This follows from the fact that
      the simpler description $\<\stmt_1',\ell,e,\theta'>$ is
      nested-irreducible:
      \begin{enumerate*}[label=(\roman*)]
      \item if $e\ne\nil$ then $\delta'_\Hnst$;
      \item if $\stmt_1'=\ceu{\Nop}$ or $\stmt_1'=\ceu{\Break}$ then the
        rest of the proof is similar to the cases for
        $\stmt=\ceu{\Nop;\stmt''}$ and $\stmt=\ceu{\Break;\stmt''}$; and
      \item if $\isblk(\stmt_1',\ell)$ then, by definition,
        $\isblk(\stmt_1';\stmt_2,\ell)$, and so $\delta'_\Hnst$.
      \end{enumerate*}
    \end{case}
    %%
  \item[{[$\ceu{\Loop{\stmt'}}$]}]
    By \R{loop-expd},
    \[
      \<\ceu{\Loop{\stmt'}},\nil,\theta>
      \nst[1]\<\ceu{\stmt'\AtLoop\stmt'},\nil,\theta'>.
    \]
    %%
    The rest of the proof is similar to that of the case for $\ceu{\AtLoop}$.
    %% 
  \item[{[$\ceu{\stmt'\ParAnd\stmt''}$]}]  By \R{par/and-expd},
    \begin{align*}
      &\<\ceu{\stmt'\ParAnd\stmt''},\ell,\nil,\theta>\\
      &\qquad\nst[1]\<\ceu{\stmt1\AtParAnd(\RunAt(\ell);\stmt'')},
        \ell,\nil,\theta>.
    \end{align*}
    The rest of the proof is similar to that of the case for
    $\ceu{\AtParAnd}$.
    %% 
  \item[{[$\ceu{\stmt'\ParOr\stmt''}$]}] By \R{par/or-expd},
    \begin{align*}
      &\<\ceu{\stmt'\ParOr\stmt''},\ell,\nil,\theta>\\
      &\qquad\nst[1]\<\ceu{\stmt1\AtParOr(\RunAt(\ell);\stmt'')},
        \ell,\nil,\theta>.
    \end{align*}
    The rest of the proof is similar to that of the case for
    $\ceu{\AtParOr}$.
    %%
  \item[{[$\ceu{\RunAt(\ell)}$]}] By \R{run-at},
    \[
      \<\ceu{\RunAt(\ell)},\ell,\nil,\theta>
      \nst[1]\<\ceu{\Nop},\ell,\nil,\theta>_\Hnst.
    \]
    %%
  \item[{[$\ceu{\AtVar{v\,n\,\stmt'}}$]}] Trivial if $\stmt'$ is
    $\ceu{\Nop}$ or $\ceu{\Break}$.  Suppose that is not the case.  Then, by
    the induction hypothesis,
    \[
      \<\stmt',\nil,(v,n):\theta>
      \nst[*]\<\stmt'',e,(v,n'):\theta'>_\Hnst.
    \]
    And so, by \R{var-adv},
    \[
      \<\ceu{\AtVar{v\,n\,\stmt'}},\nil,\theta>
      \nst[1]\<\ceu{\AtVar{v\,n',\stmt''}},e,\theta'>_\Hnst.
    \]
    %%
  \item[{[$\ceu{\stmt'\AtLoop{\stmt''}}$]}]
    There are three subcases.
    \begin{case}
    \item[{[$stmt'=\ceu{\Nop}$]}] By \R{loop-nop} and
      \R{loop-expd},
      \begin{align*}
        \<\ceu{\Nop\AtLoop{\stmt''}},\ell,\nil,\theta>
        &\nst[1]\<\ceu{\Loop{\stmt''}},\ell,\nil,\theta>\\
        &\nst[1]\<\ceu{\stmt''\AtLoop\stmt''},\ell,\theta>
      \end{align*}
      By Assumption~\ref{ass.syn-rest.loop},
      \[
        \<\stmt'',\ell,\nil,\theta>\nst[*]\<\stmt''',\ell,e,\theta'>.
      \]
      where either~$\stmt'''=\ceu{\Break}$ or $\isblk(\stmt''',\ell)$.  By
      Lemma~\ref{lem.props-nst-i},
      \begin{align*}
        &\<\ceu{\stmt''\AtLoop\stmt''},\ell,\nil,\theta>\\
        &\qquad\nst[*]\<\ceu{\stmt'''\AtLoop\stmt''},\ell,e,\theta'>=\delta'.
      \end{align*}
      If~$\stmt'''=\ceu{\Break}$, then this case becomes
      $\ceu{\Break\AtLoop\stmt''}$.  Otherwise, $\isblk(\stmt''',\ell)$, and
      so, $\delta'_\Hnst$.
      %%
    \item[{[$stmt'=\ceu{\Break}$]}] By \R{loop-brk},
      \[
        \<\ceu{\Break\AtLoop{\stmt''}},\nil,\theta>
        \nst[1]\<\ceu{\Nop},\nil,\theta>_\Hnst.
      \]
    \item[{[$stmt'\ne\ceu{\Nop,\Break}$]}] Similar to the case for\sloppy\
      $\stmt';\stmt''$ with $\stmt'\ne\ceu{\Nop,\Break}$.
    \end{case}
    %%
  \item[{[$stmt=\ceu{\stmt'\AtParAnd\stmt''}$]}] The proof depends on
    whether or not $\stmt'$ is blocked.  We will prove the case for
    $\lnot\isblk(\stmt',\ell)$; the proof of the case for
    $\isblk(\stmt',\ell)$ is similar.  So, assuming $\stmt'$ is not blocked,
    there are three subcases.
    \begin{case}
    \item[{[$stmt'=\ceu{\Nop}$]}] By \R{par/and-nop1} and by the induction
      hypothesis, for some $\delta'$,
      \[
        \<\ceu{{\Nop}\AtParAnd{\stmt''}},\nil,\theta>
        \nst[1]\<\ceu{\stmt''},\nil,\theta>
        \nst[*]\delta'_\Hnst.
      \]
    \item[{[$stmt'=\ceu{\Break}$]}] By \R{par/and-brk1},
      \begin{align*}
        &\<\ceu{\Break\AtParAnd\stmt''},\nil,\theta>\\
        &\qquad\nst[1]\<\ceu{\clear(\stmt'');\Break},\nil,\theta>.
      \end{align*}
      By Assumption~\ref{ass.syn-rest.fin},
      \[
        \<\ceu{\clear(\stmt'')},\ell,\nil,\theta>.
        \nst[*]\<\stmt''',\ell,e,\theta'>_\Hnst.
      \]
      And so, by Lemma~\ref{lem.props-nst-i},
      \begin{align*}
        &\<\ceu{\clear(\stmt'');\Break},\ell,\nil,\theta>\\
        &\qquad\nst[*]\<\ceu{\stmt''';\Break},\ell,e,\theta'>_\Hnst.
      \end{align*}
      %%
    \item[{[$stmt'\ne\ceu{\Nop,\Break}$]}] By the induction hypothesis,
      \[
        \<\stmt',\ell,\nil,\theta>\nst[*]\<\stmt_1',\ell,e,\theta'>_\Hnst.
      \]
      Hence, by Lemma~\ref{lem.props-nst-i},
      \begin{align*}
        &\<\ceu{\stmt'\AtParAnd\stmt''},\ell,\nil,\theta>\\
        &\qquad\nst[*]\<\ceu{\stmt_1'\AtParAnd\stmt''},\ell,e,\theta>=\delta'.
      \end{align*}
      It remains to be shown that description~$\delta'$ is
      nested-irreducible or leads to a nested-irreducible description.  This
      proof is similar to that of the case for $\ceu{\stmt';\stmt''}$ with
      $\stmt'\ne\ceu{\Nop,\Break}$.  That is,
      \sloppy
      \begin{enumerate*}[label=(\roman*)]
      \item if $e\ne\nil$ then $\delta'_\Hnst$;
      \item if $\stmt_1'=\ceu{\Nop,\Break}$, then this case becomes
        $\ceu{\Nop\AtParAnd\stmt''}$ or $\ceu{\Break\AtParAnd\stmt''}$; and
      \item if $\isblk(stmt_1')$, then this case becomes
        $\ceu{\stmt'\AtParAnd\stmt''}$ with $\isblk(stmt',\ell)$ evaluating
        to true.
      \end{enumerate*}
    \end{case}
    %%
  \item[{[$stmt=\ceu{\stmt'\AtParOr\stmt''}$]}] Similar to the case
    for\sloppy\ $\ceu{\stmt'\AtParAnd\stmt''}$.\qedhere
  \end{case}
\end{proof}


\lemrankout*
\begin{proof}
  Let
  \begin{alignat*}{2}
    \delta&=\<\stmt,\ell,e,\theta>      &\qquad\rank(\delta)&=(i,j)\\
    \delta'&=\<\stmt',\ell',e',\theta'> &\qquad\rank(\delta')&=(i',j')
  \end{alignat*}
  Then
  \begin{enumerate}
  \item Suppose~$\delta\outpush\delta'$.  Then, by \R{push}, $e\ne\nil$,
    $\stmt'=\bcast(\stmt,e)$, $\ell'=\ell+1$, $e'=\nil$, and
    $\theta=\theta'$.  By definition of $\rank$, $j=\ell+1$, as~$e\ne\nil$,
    and~$j'=\ell'=\ell+1$.  Hence, $j=j'$.  It remains to be shown
    that~$i=i'$:
    \begin{align*}
      i&=\pot(\stmt,e)
         \tag*{by definition of $\rank$}\\
       &=\pot'(\bcast(\stmt,e))
         \tag*{by defintion of $\pot$}\\
       &=\pot'(\stmt')
         \tag*{as~$\stmt'=\bcast(\stmt,e)$}\\
       &=\pot'(\bcast(\stmt',e'))
         \tag*{as~$e'=\nil$ (nothing changes)}\\
       &=\pot(\stmt',e')
         \tag*{by definition of $\pot$}\\
       &=i'
         \tag*{by definition of $\rank$}
    \end{align*}
    Therefore, $(i,j)=(i',j')$.

  \item Suppose~$\delta\outpop\delta'$.  Then, by~\R{pop}, $\stmt=\stmt'$,
    $\ell>0$, $\ell'=\ell-1$, $e=e'=\nil$, and $\theta=\theta'$.  By
    definition of $\pot$,
    \[
      \pot(\bcast(\stmt,e))=\pot(\bcast(\stmt',e')),
    \]
    as~$\stmt=\stmt'$ and~$e=e'=\nil$.  Hence, $i=i'$.  And by definition of
    $\rank$, $j=\ell$, as~$e=\nil$, and~$j'=\ell-1$, as~$e'=\nil$
    and~$\ell'=\ell-1$.  Hence, $j>j'$.  Therefore, $(i,j)>(i',j')$.\qedhere
  \end{enumerate}
\end{proof}


\lemranknst*
\begin{proof}
  By induction on the structure of $\nst$ derivations.  In this proof, we
  will write~$\pot_e(\stmt)$ for~$\pot(\stmt,e)$, and we will often use
  (implicitly) the fact that~$\pot_\nil=\pot'$, i.e, broadcasting the empty
  event is a no-op ($\bcast(\stmt,\nil)=\stmt$).

  Let
  \begin{alignat*}{2}
    \delta&=\<\stmt,\ell,e,\theta>      &\qquad\rank(\delta)&=(i,j)\\
    \delta'&=\<\stmt',\ell',e',\theta'> &\qquad\rank(\delta')&=(i',j')
  \end{alignat*}
  By the hypothesis of the lemma, there is derivation~$\pi$ such that
  \[
    \AxiomC{\strut$\vdots$}
    \LeftLabel{\strut$\pi=\null$}
    \UnaryInfC{\strut$\delta\nst\delta'$}
    \DisplayProof
  \]
  By definition of~$\nst$, $e=\nil$ and~$\ell=\ell'$.  Depending on the
  structure of program~$\stmt$, the following cases are possible.  In each
  one of them we show that $\rank(\delta)\ge\rank(\delta')$.

  \begin{case}
  \item[{[$\ceu{\Var{v\,\stmt_1}}$]}] Then $\pi$ is an instance of
    \R{var-expd}.  Hence, $\stmt'=\ceu{\Var{v\,\bot\,\stmt_1}}$ and
    $e'=\nil$.  Thus,
    \begin{align*}
      \rank(\delta)
      &=(\pot_\nil(\ceu{\Var{v\,\stmt_1}}),\ell)\\
      &=(\pot'(\stmt_1),\ell)\\
      &=(\pot_\nil(\stmt_1),\ell)=\rank(\delta').
    \end{align*}
    %%
  \item[{[$\ceu{v\coloneqq\expr}$]}] Then $\pi$ is an instance of
    \R{assign}. Hence, $\stmt'=\ceu{\Nop}$ and $e'=\nil$.  Thus,
    $\rank(\delta)=\rank(\delta')=(0,\ell)$.
    %%
  \item[{[$\ceu{\EmitInt(e_1)}$]}] Then $\pi$ is an instance of
    \R{emit-int}.  Hence, $\stmt'=\ceu{\RunAt(\ell)}$ and $e'=e_1\ne\nil$.
    Thus,
    \[
      \rank(\delta)=(1,\ell)>(0,\ell+1)=\rank(\delta').
    \]
    %%
  \item[{[$\ceu{\IfElse{\expr}{\stmt_1}{\stmt_2}}$]}] Suppose
    $\eval(\theta,\expr)\ne0$.  (The case for $\eval(\theta,\expr)=0$ is
    similar.)  Then $\pi$ is an instance of \R{if-true}.  Hence,
    $\stmt'=\ceu{\stmt_1}$ and $e'=\nil$.
    \begin{align*}
      \rank(\delta)&=(\pot_\nil(\ceu{\IfElse{\expr}{\stmt_1}{\stmt_2}},\ell)\\
                   &=(\max\{pot'(\stmt_1),pot'(\stmt_2)\},\ell)\\
                   &\ge(\pot'(\stmt_1),\ell)\\
                   &=\pot_\nil(\stmt_1,\ell)=\rank(\delta').
    \end{align*}
    %%
  \item[{[$\ceu{\stmt_1;\stmt_2}$]}]  There are three subcases.
    \begin{case}
    \item [{[$\stmt_1=\ceu{\Nop}$]}] Then $\pi$ is an instance of
      \R{seq-nop}.  Hence, $\stmt'=\stmt_2$ and $e'=\nil$.  Thus,
      \begin{align*}
        \rank(\delta)&=(\pot_\nil(\ceu{\Nop;\stmt_2}),\ell)\\
                     &=(0+pot'(\stmt_2),\ell)\\
                     &=(\pot_\nil(\stmt_2)=\rank(\delta').
      \end{align*}
      %%
    \item [{[$\stmt_1=\ceu{\Break}$]}] Then $\pi$ is an instance of
      \R{seq-brk}.  Hence, $\stmt'=\ceu{\Break}$ and $e'=\nil$.  Thus,
      $\rank(\delta)=\rank(\delta')=(0,\ell)$.
      %%
    \item [{[$\stmt_1\ne\ceu{\Nop,\Break}$]}] Then $\pi$ is an instance of
      \R{seq-adv}.  Hence, there is a derivation $\pi'$ such that
      \[
        \AxiomC{\strut$\vdots$}
        \LeftLabel{\strut$\pi'=\null$}
        \UnaryInfC{\strut$\<\stmt_1,\ell,\nil,\theta>
          \nst\<\stmt_1',\ell,e',\theta'>$}
        \UnaryInfC{\strut$\<\stmt_1;\stmt_2,\ell,\nil,\theta>
          \nst\<\stmt_1';\stmt_2,\ell,e',\theta'>$}
        \DisplayProof
      \]
      By the induction hypothesis,
      \begin{equation}
        \label{lem.rank-nst.seq-adv.eq1}
        \rank\<\stmt_1,\ell,\nil,\theta>
        \ge\rank\<\stmt_1',\ell,e',\theta'>\,.
      \end{equation}
      There are two possibilities: either $e'=\nil$ or $e'\ne\nil$.

      If~$e'=\nil$, then
      \begin{align*}
        \rank(\delta)&=(\pot_\nil(\stmt),\ell)\\
                     &=(\pot'(\stmt_1)+\pot'(\stmt_2),\ell)\\
        \shortintertext{and}
        \rank(\delta')&=(\pot_\nil(\stmt'),\ell)\\
                     &=(\pot'(\stmt_1')+\pot'(\stmt_2),\ell)
      \end{align*}
      By~\eqref{lem.rank-nst.seq-adv.eq1},
      $\pot'(\stmt_1)\ge\pot'(\stmt_1')$.
      Thus,
      \[
        \rank(\delta)\ge\rank(\delta')\,.
      \]

      If however $e'\ne\nil$, then $\pi'$ contains one application of
      \R{emit-int}, which consumes at least one reachable $\ceu{\EmitInt}$
      statement from $\stmt_1$, and so $\pot'(\stmt_1)>\pot'(\stmt_1')$.
      Thus,
      \begin{align*}
        \rank(\delta)&=(\pot_\nil(\stmt),\ell)\\
                     &=(\pot'(\stmt_1)+\pot'(\stmt_2),\ell)\\
                     &>(\pot'(\stmt_1')+\pot'(\stmt_2),\ell+1)\\
                     &=(\pot_e(\stmt'),\ell+1)=\rank(\delta').
      \end{align*}
      Because~$e$ is necessarily an internal event, as it was generated by
      an $\ceu{\EmitInt}$, broadcasting it to $\stmt'$ cannot modify the
      potency of the resulting statement.
    \end{case}
    %%
  \item[{[$\ceu{\Loop{\stmt_1}}$]}] Then $\pi$ is an instance of
    \R{loop-expd}.  Hence, $\stmt'=\ceu{\stmt_1\AtLoop\stmt_1}$ and
    $e'=\nil$.  By the syntactic restriction for loop bodies, all execution
    paths within $\stmt_1$ contain a matching $\ceu{\Break}$ or
    $\ceu{\AwaitExt}$ or $\ceu{\EveryExt}$.  Thus,
    \[
      \rank(\delta)=\rank(\delta')=(\pot_\nil(\stmt_1),\ell).
    \]
    %%
  \item[{[$\ceu{\stmt_1\ParAnd\stmt_2}$]}] Then $\pi$ is an instance of
    \R{par/and-expd}.  Hence,
    $\stmt'=\ceu{\stmt_1\AtParAnd(\RunAt(\ell);\stmt_2)}$ and $e'=\nil$.
    Thus,
    \begin{align*}
      \rank(\delta)&=(\pot_\nil(\ceu{\stmt_1\ParAnd\stmt_2}),\ell)\\
                   &=(\pot'(\stmt_1)+\pot'(\stmt_2),\ell)\\
                   &\ge(\pot_\nil(\stmt'),\ell)=\rank(\delta').
     \end{align*}
    %%
   \item[{[$\ceu{\stmt_1\ParOr\stmt_2}$]}] Similar to the case for
     $\ceu{\ParAnd}$.
    %%
  \item[{[$\ceu{\RunAt(\ell)}$]}] Then $\pi$ is an instance of \R{run-at}.
    Hence, $\stmt'=\ceu{\Nop}$ and $e'=\nil$.  Thus,
    $\rank(\delta)=\rank(\delta')=(0,\ell)$.
    %%
  \item[{[$\ceu{\AtVar{v\,n\,\stmt_1}}$]}] Similar to the case for
    sequences.
    %%
  \item[{[$\ceu{\stmt_1\AtLoop{\stmt_2}}$]}]
    There are three subcases.
    \begin{case}
    \item [{[$\ceu{\stmt_1=\ceu{\Nop}}$]}] Then $\pi$ is an instance of
      \R{loop-nop}.  Hence, $\stmt'=\ceu{\Loop\stmt_2}$ and $e'=\nil$.
      Thus,
      \begin{align*}
        \rank(\delta)&=(\pot_\nil(\ceu{\Nop\AtLoop\stmt_2}),\ell)\\
                     &=(0+\pot'(\stmt_2),\ell)\\
                     &=(\pot_\nil(\stmt_2),\ell)=\rank(\delta').
      \end{align*}
    \item [{[$\ceu{\stmt_1=\ceu{\Break}}$]}] Then $\pi$ is an instance of
      \R{loop-brk}.  Hence, $\stmt'=\ceu{\Nop}$ and $e'=\nil$.
      Thus,
      \begin{align*}
        \rank(\delta)&=(\pot_\nil(\ceu{\Break\AtLoop\stmt_2}),\ell)\\
                     &=(0,\ell)\\
                     &=(\pot'(\ceu{\Nop}),\ell)=\rank(\delta').
      \end{align*}
    \item [{[$\ceu{\stmt_1\ne\ceu{\Nop,\Break}}$]}] Then $\pi$ is an
      instance of \R{loop-adv}.  Hence, there is a derivation $\pi'$ such
      that
      \[
        \AxiomC{\strut$\vdots$}
        \LeftLabel{\strut$\pi'=\null$}
        \UnaryInfC{$\<\stmt_1,\ell,\nil,\theta>
          \nst\<\stmt_1',\ell,e',\theta'>$}
        \UnaryInfC{$\<\stmt,\ell,\nil,\theta>
          \nst\<\ceu{\stmt_1'\AtLoop\stmt_2},\ell,e',\theta'>$}
        \DisplayProof
      \]
      There are two possibilities: either
      \begin{enumerate*}[label=(\roman*)]
      \item every execution path in $\stmt_1$ contains a matching
        $\ceu{\Break}$, $\ceu{\AwaitExt}$, or $\ceu{\EveryExt}$; or
      \item some execution path in $\stmt_1$ does not contain a matching
        $\ceu{\Break}$ or $\ceu{\AwaitExt}$ or $\ceu{\EveryExt}$.
      \end{enumerate*}
      
      If~(i) holds, then $\pot'(\stmt)=\pot'(\stmt_1)$.  Moreover, a single
      $\nst$ cannot terminate the loop, as by hypothesis
      $\stmt_1\ne\ceu{\Break}$, nor can it consume an $\ceu{\AwaitExt}$ or
      $\ceu{\EveryExt}$ (which could increase $\pot'(\stmt_1')$).  This
      means that all execution paths in $\stmt_1'$ still contain a matching
      $\ceu{\Break}$, $\ceu{\AwaitExt}$ or $\ceu{\EveryExt}$.  And so,
      \[
        \pot'(\stmt')=\pot'(\stmt_1').
      \]

      If~(ii) holds then, since by hypothesis $\stmt_1\ne\ceu{\Nop}$, a
      single $\nst$ cannot restart the loop which means that $\stmt_1'$
      still contain some execution path in which a matching $\ceu{\Break}$,
      $\ceu{\AwaitExt}$, or $\ceu{\EveryExt}$ does not occur.  And so,
      \[
        \pot'(\stmt')=\pot'(\stmt_1')+\pot'(\stmt_2).
      \]

      In either case, the rest of the proof is similar to that of the case
      for sequences $\stmt_1;\stmt_2$ with~$\stmt_1\ne\ceu{\Nop,\Break}$.
    \end{case}
    %%
  \item[{[$\ceu{\stmt_1\AtParAnd\stmt_2}$]}] Similar to the case for
    $\ceu{\AtParOr}$.
    %%
  \item[{[$\ceu{\stmt_1\AtParOr\stmt_2}$]}] The proof depends on whether or
    not $\stmt_1$ is blocked.  We will prove the case for
    $\isblk(\stmt_1,\ell)$; the proof of the case for
    $\lnot\isblk(\stmt_1,\ell)$ is similar.  So, assuming $\stmt_1$ is
    blocked, there are three subcases.
    \begin{case}
    \item[{[$\stmt_2=\ceu{\Nop}$]}] Then $\pi$ is an instance of
      \R{par/or-nop2}.  Hence, $\stmt'=\clear(\stmt_1)$ and $e'=\nil$.
      Thus,
      \begin{align*}
        \rank(\delta)&=(\pot_\nil(\ceu{\stmt_1\AtParOr\Nop}),\ell)\\
                     &=(\pot'(\stmt_1)+0,\ell)\\
                     &\ge(\pot'(\clear(\stmt_1)),\ell)\\
                     &=(\pot_\nil(\clear(\stmt_1)),\ell)=\rank(\delta').
      \end{align*}
      Since it only returns the body of \emph{active} $\ceu{\Fin}$
      statements (i.e., statements which are already being considered by
      $\pot$), function $\clear$ cannot increase the potency of the
      resulting statement.
      %%
    \item[{[$\stmt_2=\ceu{\Break}$]}] Then $\pi$ is an instance of
      \R{par/or-brk2}.  Hence, $\stmt'=\ceu{\clear(\stmt_1);\Break}$ and
      $e'=\nil$.  Thus,
      \begin{align*}
        \rank(\delta)&=(\pot_\nil(\ceu{\stmt_1\AtParOr\Break}),\ell)\\
                     &=(\pot'(\stmt_1)+0,\ell)\\
                     &\ge(\pot'(\ceu{\clear(\stmt_1);\Break}),\ell)\\
                     &=(\pot_\nil(\ceu{\clear(\stmt_1);\Break}),\ell)
                       =\rank(\delta').
      \end{align*}
      %%
    \item[{[$\stmt_2\ne\ceu{\Nop,\Break}$]}] Then $\pi$ is an instance of
      \R{par/or-adv2}.  The rest of the proof is similar to that of the case
      for sequences $\stmt_1;\stmt_2$ with~$\stmt_1\ne\ceu{\Nop,\Break}$.
      \qedhere
    \end{case}
  \end{case}
\end{proof}


\thmranknstx*
\begin{proof}
  If $\delta\nst[*]\delta'$ then $\delta\nst[i]\delta'$, for some~$i$.  We
  proceed by induction on~$i$.  The theorem is trivially true for $i=0$ and
  follows directly from Lemma~\ref{lem.rank-nst} for $i=1$.  Suppose
  \[
    \delta\nst[1]\delta_1'\nst[i-1]\delta',
  \]
  for some~$i>1$.  Then, by Lemma~\ref{lem.rank-nst} and by the induction
  hypothesis,
  \[
    \rank(\delta)\ge\rank(\delta_1')\ge\rank(\delta')\,.\qedhere
  \]
\end{proof}

% LocalWords:  expd nop brk stmt irr nst eq TODO

%%% Local Variables:
%%% mode: latex
%%% TeX-master: "index.tex"
%%% End:

