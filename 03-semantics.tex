% \newcommand{\NST}{\1\xrightarrow[\mathit{nst}]\1}
% \newcommand{\OUT}{\1\xrightarrow[\mathit{out}]\1}
% \newcommand{\LL}{\langle}
% \newcommand{\RR}{\rangle}
% \newcommand{\DS}{\displaystyle}

% \newcommand{\1}{\;}
% \newcommand{\2}{\;\;}
% \newcommand{\3}{\;\;\;}
% \newcommand{\5}{\;\;\;\;\;}

\def\JOT{.8\jot}

\section{Formal Semantics of \CEU}
\label{sec.sem}

In this section, we introduce a reduced, abstract syntax for \CEU and
present an operational semantics that formalizes the behavior of its
programs.
% We describe a small synchronous kernel highlighting the peculiarities of \CEU,
% in particular, the stack-based execution for internal events.
%
The semantics deals only with the control aspects of \CEU.
Side-effects and system calls are encapsulated in a memory access
primitive and are assumed to behave like in conventional imperative
languages.

\subsection{Abstract Syntax}
\label{sec.sem.syntax}

%-
% \begin{lstlisting}[
%   %numbers=left,
%   basicstyle=\ttfamily\footnotesize,
%   float=h,
%   caption={Reduced syntax of \CEU.},
%   label={lst.formal.syntax},
%   mathescape=true
% ]
%                                    // primary expressions
%   p ::= mem(id)                    (any memory access to `id')
%       $|$ awaitExt(id)               (await external event `id')
%       $|$ awaitInt(id)               (await internal event `id')
%       $|$ emitInt(id)                (emit internal event `id')
%       $|$ break                      (loop escape)
%                                    // compound expressions
%       $|$ if mem(id) then p else p   (conditional)
%       $|$ p ; p                      (sequence)
%       $|$ loop p                     (repetition)
%       $|$ every id p                 (event iteration)
%       $|$ p and p                    (par/and)
%       $|$ p or p                     (par/or)
%       $|$ fin p                      (finalization)
%                                    // derived by semantic rules
%       $|$ p @loop p                  (unwinded loop)
%       $|$ p @and q                   (unwinded par/and)
%       $|$ p @or q                    (unwinded par/or)
%       $|$ @canrun(n)                 (can run on stack level `n')
%       $|$ @nop                       (terminated expression)
% \end{lstlisting}
%-

The grammar below defines the syntax of a subset of \CEU that is
sufficient to describe all peculiarities of the language.
\bgroup
\def\lbl#1{\enspace\text{\emph{#1}}}%
\newdimen\X
\X=-1.3\jot
\begin{alignat*}{2}
  P\Coloneqq
      &\enspace\ceu{\Mem(\Id)}
      &&\lbl{any memory access to~``$\ceu\Id$''}\\[\X]
      %%
  \mid&\enspace\ceu{\AwaitExt(\Id)}
      &&\lbl{await external event~``$\ceu\Id$''}\\[\X]
      %%
  \mid&\enspace\ceu{\AwaitInt(\Id)}
      &&\lbl{await internal event~``$\ceu\Id$''}\\[\X]
      %%
  \mid&\enspace\ceu{\EmitInt(\Id)}
      &&\lbl{emit internal event~``$\ceu\Id$''}\\[\X]
      %%
  \mid&\enspace\ceu{\Break}
      &&\lbl{loop escape}\\[\X]
      %%
  \mid&\enspace\ceu{\IfElse{\Mem(\Id)}{P_1}{P_2}}
      &&\lbl{conditional}\\[\X]
      %%
  \mid&\enspace\ceu{P_1\,;\,P_2}
      &&\lbl{sequence}\\[\X]
      %%
  \mid&\enspace\ceu{\Loop P_1}
      &&\lbl{repetition}\\[\X]
      %%
  \mid&\enspace\ceu{\Every{\Id}\ P_1}
      &&\lbl{event iteration}\\[\X]
      %%
  \mid&\enspace\ceu{P_1\And P_2}
      &&\lbl{par/and}\\[\X]
      %%
  \mid&\enspace\ceu{P_2\Or P_2}
      &&\lbl{par/or}\\[\X]
      %%
  \mid&\enspace\ceu{\Fin P}
      &&\lbl{finalization}\\[\X]
      %%
  \mid&\enspace\ceu{P_1\AtLoop P_2}
      &&\lbl{unwinded loop}\\[\X]
      %%
  \mid&\enspace\ceu{P_1\AtAnd\ P_2}
      &&\lbl{unwinded par/and}\\[\X]
      %%
  \mid&\enspace\ceu{P_1\AtOr\ P_2}
      &&\lbl{unwinded par/or}\\[\X]
      %%
  \mid&\enspace\ceu{\CanRun(n)}
      &&\lbl{can run on stack level~$n$}\\[\X]
      %%
  \mid&\enspace\ceu{\Nop}
      &&\lbl{terminated program}
\end{alignat*}
\egroup

The~$\ceu{\Mem(id)}$ primitive represents a read or write to the memory
location identified by~$id$.\footnotemark\
%
Following the synchronous hypothesis, $\ceu{\Mem}$ statements and
expressions are considered to be atomic and instantaneous.
%
% As the challenging parts of \CEU reside on its control structures, we are not
% concerned here with a precise semantics for side effects, but only with their
% occurrences in programs.
%
%The special notation $nop$ is used to represent an innocuous $mem$ expression
%(it can be thought as a synonym for $mem(\epsilon)$, where $\epsilon$ is an
%unused identifier).
%
\footnotetext{Although the same symbol~$\ceu{\Id}$ is used in their
  definition, $\ceu{\Mem}$, $\ceu{\AwaitExt}$, $\ceu{\AwaitInt}$
  and~$\ceu{\EmitInt}$ do not share identifiers: any identifier is either a
  variable, an external event, or an internal event.}

Most statements in the abstract language are mapped to their counterparts in
the concrete language.  The exceptions are the finalization
block~$\ceu{\Fin{p}}$ and the \texttt{@}-statements which do not exist in
the concrete language.  These result from the expansion of the transition
rules to be presented.

A further difference between the concrete and abstract languages regards the
emit-await pair.  In the concrete language, the \code{await} can be used as
an expression which evaluates to the value stored in the corresponding emitted
event, e.g., an
``\code{emit a(10)}'' awakes a ``\code{v=await a}'' setting
variable~\code{v} to~10.  Although the abstract $\ceu{\AwaitInt}$
and~$\ceu{\EmitInt}$ do not support such communication of values, it can be
easily simulated: one can use a shared variable to hold the value of an
$\ceu{\EmitInt}$ and access it after the corresponding $\ceu{\AwaitInt}$
awakes.

Finally, a ``\code{finalize $A$ with $B$ end; $C$}'' in the concrete
language is equivalent to ``\ceu{A;\;((\Fin{B})\ \Or\ C)}'' in the abstract
language.  In the concrete language, $A$ and~$C$ execute in sequence, and
the finalization code~$B$ is implicitly suspended waiting for~$C$
to terminate.  In the abstract language, ``$\ceu{\Fin B}$'' suspends forever
when reached (it is an awaiting statement that never awakes).  Hence, we
need an explicit \code{or} to execute~$C$ in parallel, whose termination
aborts ``$\ceu{\Fin B}$'', which finally causes~$B$ to execute (by the
semantic rules below).


\subsection{Operational Semantics}
\label{sec.sem.opsem}

The operational semantics is a mathematical model that describes how an
abstract \CEU\ programs reacts to a single external input event, i.e., how
starting from this input event the program advances its state until all its
trails are blocked waiting for a another input event.
%%
The formalism we use here is that of small-step operational
semantics~\cite{Plotkin-G-D-1981} in which the meaning of programs is
defined in terms of transitions of an abstract machine.  Each transition
transforms a triple consisting of a program~$p$, a stack level~$n$, and an
emitted event~$e$ into a possibly different triple, i.e.,
\[
  \<p,n,e>\trans\<p',n',e'>\,,
\]
where~$p,p'\in\P$ are abstract-language programs, $n,n'\in\N$ are non-negative
integers representing the current stack level, and~$e,e'\in\E\cup\{\nil\}$ are
the events emitted before and after the transition (both possibly being the
empty event~$\nil$).
%$\E$ is a set of primitive events.

We refer to the triples on the left-hand and right-hand sides of
symbol~$\trans$ as \emph{descriptions} (denoted~$\delta$).  The description
on the left of~$\trans$ is called the \emph{input description}, the one on
its right is called the \emph{output description}.

%-
% \begin{align*}
% p, p' &\in\P
%     && (program~as~described~in~Listing~\ref{lst.formal.syntax})
% \\
% n, n' &\in\N
%     && (current~stack~level)
% \\
% e, e' &\in\E \cup \{\epsilon\}
%     && (emitted~event,~possibly~none)
% \end{align*}
%-

At the beginning of a reaction to an input event~$id$, the input description
is initialized with stack level~0 ($n=0$) and with the externally emitted
event~$e=\ceu{\Id}$.  At the end of a reaction, after an arbitrary but
finite number of transitions, the last output description will block with a
possibly modified program~$p'$ at stack level~0 and no event
emitted\footnote{We write~$\trans[i]$ to mean~$i$ transitions in sequence,
  and we write~$\trans[*]$ to mean a finite (possibly zero) number of
  transitions in sequence.}:
\[
  \<p,0,e>\mathbin{\trans[*]}\<p',0,\nil>\,.
\]

We now proceed to give the rules for possible the transitions.  We
distinguish between two types of transitions:  \emph{outermost transitions}
$\out$ and \emph{nested transitions} $\nst$\,.

\subsubsection{Outermost Transitions}

The rules \R{push} and \R{pop} for~$\out$ transitions are non-recursive
definitions that apply to the program as a whole.  These are the only rules
that manipulate the stack level.
\begin{gather*}
  \AxiomC{$e\ne\nil$}
  \UnaryInfC{$\<p,n,e>\out\<\bcast(p,e),n+1,\nil>$}
  \DisplayProof
  \Rtag{push}\\[1.2\jot]
  %%
  \AxiomC{$n>0$}
  \AxiomC{$p=\ceu{\Nop} \vee p=\ceu{\Break} \vee \isblocked(p,n)$}
  \BinaryInfC{$\<p,n,\nil>\out\<p,n-1,\nil>$}
  \DisplayProof
  \Rtag{pop}
\end{gather*}

%-
% { \setlength{\jot}{20pt}
% \begin{eqnarray*}
% & \frac
%     { \DS e \neq \epsilon }
% %   -----------------------------------------------------------
%     { \DS \LL p,n,e \RR \OUT \LL bcast(p),n+1,\epsilon \RR }
%     & \textbf{(push)}   \\
% %%%
% & \frac
%     { \DS n>0, \2 ((p=@nop) \vee isblocked(n,p)) }
% %   -----------------------------------------------------------
%     { \DS \LL p,n,\epsilon \RR \OUT \LL p,n-1,\epsilon \RR }
%     & \textbf{(pop)}    \\
% %%%
% %& \LL p,0,\epsilon \RR \1\xrightarrow\1 \bot
%     %& \textbf{(end)}    %\\
% \end{eqnarray*}
% }
%-

Rule \R{push} can be applied whenever there is a nonempty event in the input
description,
and instantly broadcasts the event to the program, which means
    (i)~awaking any active $\ceu{\AwaitExt}$ or $\ceu{\AwaitInt}$ statements
    (see $\bcast$ in
        Figure~\ref{fig.bcast}),
    (ii)~creating a nested reaction by increasing the stack level, and, at the same time,
    (iii)~consuming the event ($e$ becomes~$\nil$).
%
Rule \R{push} is the only rule that matches an
emitted event and also immediately consumes it.

Rule \R{pop} decreases the stack level by one and can only be applied if the
program is blocked (see $\isblocked$
in Figure~\ref{fig.isblocked}) or terminated ($p=\ceu{\Nop}$ or $p=\ceu{\Break}$).
This condition ensures that an $\ceu{\EmitInt}$ only resumes after its internal
reaction completes and blocks in the current stack level.

At the beginning of a reaction, an external event is emitted, which
triggers rule \R{push}, immediately raising the stack level
to~1.
At the end of the reaction, the program will block or terminate and
successive applications of
rule~\R{pop} will lead to a description with this
same program at stack level~0.

\subsubsection{Nested Transitions}

The~$\nst$ rules are recursive definitions of the form
\[
\<p,n,\nil>\nst\<p',n,e>.
\]
%
%-
% \begin{align*}
% \LL p, n,\epsilon \RR &\NST
% \LL p',n,e        \RR
%     & \textbf{(rule-inner)}
% \end{align*}
%-
%
Nested transitions do not affect the stack level and never have an emitted
event as a precondition.  The distinction between~$\out$ and~$\nst$ prevents
rules \R{push} and \R{pop} from matching and, consequently, from
inadvertently modifying the current stack level before the nested reaction
is complete.

A complete reaction consists of a series of transitions% of the form
\begin{align*}
  \<p,0,e_\ext>\outpush\<p_1,1,\nil>
  \Big[\null\nst[*]\null\out\null\Big]\!\!\ast
  \null\nst[*]\null\outpop\<p',0,\nil>\,,
\end{align*}
%
%-
% \begin{align*}
% a) &\5\5
%     \LL p,0,ext \RR
%         \1\xrightarrow[out]{push}\1
%     \LL q,1,\epsilon \RR
% \\
% b) &\5\5 \1[ \1\xrightarrow[in]{*}\1
%     \LL r,i,e \RR
%         \1\xrightarrow[out]\1
%     \LL s,j,\epsilon \RR \1]*
% \\
% c) &\5\5 \1\xrightarrow[in]{*}\1
%     \LL t,k,\epsilon \RR
%         \1\xrightarrow[out]{pop}\1
%     \LL u,0,\epsilon \RR
% \end{align*}
%-
%
First, a~$\outpush$ starts a nested reaction at level~1.  Then, a series of
alternations between zero or more~$\nst$ transitions (nested reactions) and
a single~$\out$ transition (stack operation) takes place.  Finally, a
last~$\outpop$ transition decrements the stack level to~0 and terminates the
reaction.

The~$\nst$ rules for atoms are defined as follows:
\begingroup
\def\JOT{-.1\jot}
\begin{align*}
  \<\ceu{\Mem(\Id)},n,\nil>
  &\nst\<\ceu{\Nop},n,\nil>\Rtag{mem}\\[\JOT]
  %%
  \<\ceu{\EmitInt(\Id)},n,\nil>
  &\nst\<\ceu{\CanRun(n)},n,\ceu{\Id}>\Rtag{emit-int}\\[\JOT]
  %%
  \<\ceu{\CanRun(n)},n,\nil>
  &\nst\<\ceu{\Nop},n,\nil>\Rtag{can-run}
\end{align*}
\endgroup

%-
% { \setlength{\jot}{20pt}
% \begin{align*}
% \LL mem(id), n, \epsilon \RR &\NST
% \LL @nop, n, \epsilon \RR
%     & \textbf{(mem)}        \\
% %%%
% \LL emit(id), n, \epsilon \RR &\NST
% \LL @canrun(n), n, id \RR
%     & \textbf{(emitInt)}    \\
% %%%
% \LL @canrun(n), n, \epsilon \RR &\NST
% \LL @nop, n, \epsilon \RR
%     & \textbf{(canrun)}     \\
% \end{align*}
% }
%-

A $\ceu{\Mem}$ operation becomes a $\ceu{\Nop}$ which indicates the memory
access (rule \R{mem}).
An $\ceu{\EmitInt(id)}$ generates an event $\ceu{\Id}$ and becomes a
$\ceu{\CanRun(n)}$ which can only resume at level~$n$ (rule \R{emit-int}).
Since all~$\nst$ rules can only be applied if $e=\nil$, an $\ceu{\EmitInt}$
inevitably causes rule \R{push} to execute at the outer level, creating a new
level~$n+1$ on the stack.
Also, with the new stack level, the resulting $\ceu{\CanRun}(n)$ itself cannot
transition yet (rule~\R{can-run}), providing the desired stack-based semantics for
internal events.

The rules for conditionals and sequences are the following:
\vskip-.6\baselineskip
\begingroup
\def\JOT{-.1\jot}
\begin{gather*}
  \AxiomC{$\eval(\ceu{\Mem(\Id)})$}
  \UnaryInfC{$\<\ceu{\IfElse{\Mem(\Id)}{p}{q}},n,\nil>\nst\<p,n,\nil>$}
  \DisplayProof
  \Rtag{if-true}\\[\JOT]
  %%
  \AxiomC{$\lnot\eval(\ceu{\Mem(\Id)})$}
  \UnaryInfC{$\<\ceu{\IfElse{\Mem(\Id)}{p}{q}},n,\nil>\nst\<q,n,\nil>$}
  \DisplayProof
  \Rtag{if-false}\\[\JOT]
  %%
  \AxiomC{$\<p,n,\nil>\nst\<p',n,e>$}
  \UnaryInfC{$\<\ceu{p\,;\,q},n,\nil>\nst\<\ceu{p';\,q},n,e>$}
  \DisplayProof
  \Rtag{seq-adv}\\[\JOT]
  \<\ceu{\Nop;\,q},n,\nil>\nst\<q,n,\nil>\Rtag{seq-nop}\\[\JOT]
  %%
  \<\ceu{\Break;\,q},n,\nil>\nst\<\ceu{\Break},n,\nil>\Rtag{seq-brk}
\end{gather*}
\endgroup
% \begin{align*}
%   \<\ceu{\Nop;\,q},n,\nil>&\nst\<q,n,\nil>\Rtag{seq-nop}\\[\JOT]
%   %%
%   \<\ceu{\Break;\,q},n,\nil>&\nst\<\ceu{\Break},n,\nil>\Rtag{seq-brk}
% \end{align*}

%-
% { \setlength{\jot}{20pt}
% \begin{eqnarray*}
% & \frac
%     { \DS val(id) \neq 0 }
% %   -----------------------------------------------------------
%     { \DS \LL (if~mem(id)~then~p~else~q),n,\epsilon \RR \NST
%           \LL p, n, \epsilon \RR }
%     & \textbf{(if-true)}       \\
% %%%
% & \frac
%     { \DS val(id,n) = 0 }
% %   -----------------------------------------------------------
%     { \DS \LL (if~mem(id)~then~p~else~q),n,\epsilon \RR \NST
%           \LL q,n,\epsilon \RR }
%     & \textbf{(if-false)}       \\
% %%%
% & \frac
%     { \DS \LL p,n,\epsilon \RR \NST \LL p',n,e \RR }
% %   -----------------------------------------------------------
%     { \DS \LL (p~;~q), n, \epsilon \RR \NST \LL (p'~;~q), n, e \RR }
%     & \textbf{(seq-adv)}      \\
% %%%
% & \LL (@nop~;~q),n,\epsilon \RR \NST  \LL q,n,\epsilon \RR
%     & \textbf{(seq-nop)}      \\
% %%%
% & \LL (break~;~q),n,\epsilon \RR \NST \LL break,n,\epsilon \RR
%     & \textbf{(seq-brk)}
% \end{eqnarray*}
% }
%-

Rules \R{if-true} and \R{if-false} are the only rules that use~$\ceu{\Mem}$
in a way that affects the control flow.
%
Function~$\eval$ evaluates a~$\ceu{\Mem}$ expression to a boolean value.
%
%Although the value here is arbitrary, it is unique in a reaction, because a
%given expression can execute only once within it (remember that $loops$ must
%contain $awaits$ which, from rule \textbf{await}, cannot awake in the same
%reaction they are reached).
%For all other rules, we omit these values (e.g., \textbf{seq-nop}).

%As determined for nested rules, compound expressions also can only have
%$\epsilon$ as a precondition and they never modify $n$.
%However, they can still emit an event to nest another reaction.
%For instance, in rule \textbf{seq-adv}, if the sub-expression $p$ emits event
%$e$, the whole composition also emits $e$.
%However, rules \textbf{push} and \textbf{pop} can only match at the outermost
%level.

The rules for loops are similar to those for sequences, but use ``\code{@}''
as separators to bind breaks to their enclosing loops:
\begin{align*}
  \<\ceu{\Loop{p}},n,\nil>
  &\nst\<\ceu{p\AtLoop{p}},n,\nil>\Rtag{loop-expd}\\[\JOT]
  %%
  &\hskip-6.35em
  \AxiomC{$\<q,n,\nil>\nst\<q',n,e>$}
  \UnaryInfC{$\<\ceu{q\AtLoop{p}},n,\nil>\nst\<\ceu{q'\AtLoop{p}},n,e>$}
  \DisplayProof
  \Rtag{loop-adv}\\[\JOT]
  %%
  \<\ceu{\Nop\AtLoop{p}},n,\nil>
  &\nst\<\ceu{\Loop{p}},n,\nil>\Rtag{loop-nop}\\[\JOT]
  %%
  \<\ceu{\Break\AtLoop{p}},n,\nil>
  &\nst\<\ceu{\Nop},n,\nil>\Rtag{loop-brk}
\end{align*}

%-
% %
% { \setlength{\jot}{20pt}
% \begin{eqnarray*}
% & \LL (loop~p),n,\epsilon \RR \NST \LL (p~@loop~p), n, \epsilon \RR
%     & \textbf{(loop-expd)}       \\
% %%%
% & \frac
%     { \DS \LL p,n,\epsilon \RR \NST \LL p',n,e \RR }
% % -----------------------------------------------------------
%     { \DS \LL (p~@loop~q),n,\epsilon \RR \NST \LL (p'~@loop~q), n, e \RR }
%     & \textbf{(loop-adv)}    \\
% %%%
% & \LL (@nop~@loop~p), n, \epsilon \RR \NST \LL (loop~p), n, \epsilon \RR
%     & \textbf{(loop-nop)}    \\
% %%%
% & \LL (break~@loop~p), n, \epsilon \RR \NST \LL @nop, n, \epsilon \RR
%     & \textbf{(loop-brk)}
% \end{eqnarray*}
% }
%-

When a program encounters a $\ceu{\Loop}$, it first expands its body in sequence with
itself (rule \R{loop-expd}).
Rules \R{loop-adv} and \R{loop-nop} are similar to rules
\R{seq-adv} and \R{seq-nop}, advancing the loop until a~$\ceu{\Nop}$ is reached.
However, what follows the loop is the loop itself (rule \R{loop-nop}).
Note that if we used ``\code{;}'' as a separator in loops, rules
\R{loop-brk} and \R{seq-brk} would conflict.
%
Rule \R{loop-brk} escapes the enclosing loop, transforming everything into
a~$\ceu{\Nop}$.
%Rule \textbf{loop-brk} escapes the enclosing loop, transforming everything
%into a $clear(p)$.
%We cannot simply transform the loop into a $nop$ because its body may be a
%parallel composition containing finalization blocks.

The rules for~$\ceu{\And}$ and~$\ceu{\Or}$ compositions ensure that their
left branch always transition before their right
branch:
%%
\begin{gather*}
  \hskip-.9em
  \<\ceu{p\And{q}},n,\nil>
  \nst\<\ceu{p\AtAnd(\CanRun(n);q)},n,\nil>
  \Rtag{and-expd}\\[\JOT]
  %%
  \AxiomC{$\<p,n,\nil>\nst\<p',n,e>$}
  \UnaryInfC{$\<\ceu{p\AtAnd{q}},n,\nil>\nst\<\ceu{p'\AtAnd{q}},n,e>$}
  \DisplayProof
  \Rtag{and-adv1}\\[\JOT]
  %%
  \AxiomC{$\isblocked(p,n)$}
  \AxiomC{$\<q,n,\nil>\nst\<q',n,e>$}
  \BinaryInfC{$\<\ceu{p\AtAnd{q}},n,\nil>\nst\<\ceu{p\AtAnd{q'}},n,e>$}
  \DisplayProof
  \Rtag{and-adv2}\\[1.2\jot]
% \end{gather*}
% \begin{gather*}
  \<\ceu{p\Or{q}},n,\nil>
  \nst\<\ceu{p\AtOr(\CanRun(n);q)},n,\nil>
  \Rtag{or-expd}\\[\JOT]
  %%
  \AxiomC{$\<p,n,\nil>\nst\<p',n,e>$}
  \UnaryInfC{$\<\ceu{p\AtOr{q}},n,\nil>\nst\<\ceu{p'\AtOr{q}},n,e>$}
  \DisplayProof
  \Rtag{or-adv1}\\[\JOT]
  %%
  \AxiomC{$\isblocked(p,n)$}
  \AxiomC{$\<q,n,\nil>\nst\<q',n,e>$}
  \BinaryInfC{$\<\ceu{p\AtOr{q}},n,\nil>\nst\<\ceu{p\AtOr{q'}},n,e>$}
  \DisplayProof
  \Rtag{or-adv2}
\end{gather*}

%-
% { \setlength{\jot}{20pt}
% \begin{eqnarray*}
% & \LL (p~and~q),n,\epsilon \RR \NST \LL (p~@and~(@canrun(n)~;~q)),n,\epsilon \RR
%     & \textbf{(and-expd)}       \\
% %%%
% & \frac
%     { \DS \LL p,n,\epsilon \RR \NST \LL p',n,e \RR }
% %   -----------------------------------------------------------
%     { \DS \LL (p~@and~q),n,\epsilon \NST \LL (p'~@and~q),n,e \RR }
%     & \textbf{(and-adv1)}      \\
% %%%
% & \frac
%     { \DS isblocked(n,p) \1,\2 \LL q,n,\epsilon \RR \NST \LL q',n,e \RR }
% %   -----------------------------------------------------------
%     { \DS \LL (p~@and~q),n,\epsilon \RR \NST \LL (p~@and~q'), n, e \RR }
%     & \textbf{(and-adv2)}      \\
% %%%
% & \LL (p~or~q), n, \epsilon \RR \NST \LL (p~@or~(@canrun(n)~;~q)), n, \epsilon \RR
%     & \textbf{(or-expd)}       \\
% %%%
% & \frac
%     { \DS \LL p,n,\epsilon \RR \NST \LL p',n,e \RR }
% %   -----------------------------------------------------------
%     { \DS \LL (p~@or~q),n,\epsilon \RR \NST \LL (p'~@or~q), n, e \RR }
%     & \textbf{(or-adv1)}   \\
% %%%
% & \frac
%     { \DS isblocked(n,p) \1,\2 \LL q,n,\epsilon \RR \NST \LL q',n,e \RR }
% %   -----------------------------------------------------------
%     { \DS \LL (p~@or~q),n,\epsilon \RR \NST \LL (p~@or~q'), n, e \RR }
%     & \textbf{(or-adv2)}   %\\
% \end{eqnarray*}
% }
%-

Rules~\R{and-expd} and~\R{or-expd} insert a~$\ceu{\CanRun(n)}$ at the beginning
of the right branch.
This ensures that~any $\ceu{\EmitInt}$ on the left branch, which eventually becomes
a~$\ceu{\CanRun(n)}$, resumes before the right branch starts.
%
The deterministic behavior of the semantics relies on the \emph{isblocked}
predicate (see Figure~\ref{fig.isblocked}) which is used in rules
\R{and-adv2} and \R{or-adv2}.
These rules require the left branch~$p$ to be blocked for the
right branch to transition from~$q$ to~$q'$.

In a parallel~$\ceu{\AtAnd}$, if one branch terminates, the composition becomes the other branch (rules \R{and-nop1} and
\R{and-nop2} below).
%
In a parallel~$\ceu{\AtOr}$, however, if one branch terminates, the
whole composition
terminates and~$\clear$ is called to finalize the aborted
branch (rules \R{or-nop1} and \R{or-nop2}).

\begingroup
\begin{gather*}
  \<\ceu{{\Nop}\AtAnd{q}},n,\nil>\nst\<q,n,\nil>\Rtag{and-nop1}\\[\JOT]
  %%
  \AxiomC{$\isblocked(p,n)$}
  \UnaryInfC{$\<\ceu{p\AtAnd{\Nop}},n,\nil>\nst\<p,n,\nil>$}
  \DisplayProof
  \Rtag{and-nop2}\\[\JOT]
  %%
  \<\ceu{{\Nop}\AtOr{q}},n,\nil>\nst\<\clear(q),n,\nil>\Rtag{or-nop1}\\[\JOT]
  %%
  \AxiomC{$\isblocked(p,n)$}
  \UnaryInfC{$\<\ceu{p\AtOr{\Nop}},n,\nil>\nst\<\clear(p),n,\nil>$}
  \DisplayProof
  \Rtag{or-nop2}
\end{gather*}
\endgroup

%-
% { \setlength{\jot}{20pt}
% \begin{eqnarray*}
% & \LL (@nop~@and~q), n, \epsilon \RR \NST \LL q,n,\epsilon \RR
%     & \textbf{(and-nop1)}   \\
% %%%
% & \frac
%     { \DS isblocked(n,p) }
% %   -----------------------------------------------------------
%     { \DS \LL (p~@and~@nop), n, \epsilon \RR \NST \LL p,n,\epsilon \RR }
%     & \textbf{(and-nop2)}   \\
% %%%
% & \LL (@nop~@or~q), n, \epsilon \RR \NST \LL clear(q),n,\epsilon \RR
%     & \textbf{(or-nop1)}   \\
% %%%
% & \frac
%     { \DS isblocked(n,p) }
% %   -----------------------------------------------------------
%     { \DS \LL (p~@or~@nop), n, \epsilon \RR \NST \LL clear(p),n,\epsilon \RR }
%     & \textbf{(or-nop2)}   %\\
% \end{eqnarray*}
% }
%-

The~$\clear$ function (see Figure~\ref{fig.clear}) concatenates all
active~$\ceu{\Fin}$ bodies of the branch being aborted, so that they execute before the
composition rejoins.

As there are no transition rules for~$\ceu{\Fin}$ statements,
 once reached, a $\ceu{\Fin}$ halts and will only be consumed
if its trail is aborted.  At this point, its body will
execute as a result of the~$\clear$ call.  The body of a~$\ceu{\Fin}$
statement always execute within a reaction.  This is due to a syntactic
restriction: $\ceu{\Fin}$ bodies cannot
contain awaiting statements (namely, $\ceu{\AwaitExt}$, $\ceu{\AwaitInt}$,
$\ceu{\Every}$, or $\ceu{\Fin}$).

Finally, a~$\ceu{\Break}$ in one branch of a parallel escapes the closest
enclosing~$\ceu{\Loop}$, properly aborting the other branch with the~$\clear$
function:
%
\begingroup
\begin{gather*}
  \hskip-.5em
  \<\ceu{{\Break}\AtAnd{q}},n,\nil>\nst\<\ceu{\clear(q);\Break},n,\nil>
  \Rtag{and-brk1}\\[\JOT]
  %%
  \hskip-.5em
  \AxiomC{$\isblocked(p,n)$}
  \UnaryInfC{$\<\ceu{p\AtAnd{\Break}},n,\nil>
    \nst\<\ceu{\clear(p);\Break},n,\nil>$}
  \DisplayProof
  \Rtag{and-brk2}\\[\JOT]
  %%
  \<\ceu{{\Break}\AtOr{q}},n,\nil>\nst\<\ceu{\clear(q);\Break},n,\nil>
  \Rtag{or-brk1}\\[\JOT]
  %%
  \AxiomC{$\isblocked(p,n)$}
  \UnaryInfC{$\<\ceu{p\AtOr{\Break}},n,\nil>
    \nst\<\ceu{\clear(p);\Break},n,\nil>$}
  \DisplayProof
  \Rtag{or-brk2}
\end{gather*}
\endgroup

%-
% { \setlength{\jot}{20pt}
% \begin{eqnarray*}
% & \LL (break~@and~q), n, \epsilon \RR \NST \LL (clear(q)~;~break),n,\epsilon \RR
%     & \textbf{(and-brk1)}   \\
% %%%
% & \frac
%     { \DS isblocked(n,p) }
% %   -----------------------------------------------------------
%     { \DS \LL (p~@and~break), n, \epsilon \RR \NST \LL (clear(p)~;~break),n,\epsilon \RR }
%     & \textbf{(and-brk2)}   \\
% %%%
% & \LL (break~@or~q),n,\epsilon \RR \NST \LL (clear(q)~;~break),n,\epsilon \RR
%     & \textbf{(or-brk1)}   \\
% %%%
% & \frac
%     { \DS isblocked(n,p) }
% %   -----------------------------------------------------------
%     { \DS \LL (p~@or~break),n,\epsilon \RR \NST \LL (clear(p)~;~break),n,\epsilon \RR }
%     & \textbf{(or-brk2)}   %\\
% \end{eqnarray*}
% }
%-

A reaction eventually blocks in~$\ceu{\AwaitExt}$, $\ceu{\AwaitInt}$,
$\ceu{\Every}$, $\ceu{\Fin}$, and~$\ceu{\CanRun}$ statements in parallel
trails.
%
Then, if none of the trails is blocked in~$\ceu{\CanRun}$, it means that the
program cannot advance in the current reaction.
%
However, $\ceu{\CanRun}$ statements can still resume at lower stack indexes
and will eventually resume in the current reaction (see rule \R{pop}).

\begin{figure}[h]
\small
\begin{gather*}
  \boxed{
    \begin{align*}
      %%
      %%-
      \shortintertext{\llap{(i)~}Function~$\bcast$:}
      %%-
      %%
      \bcast(\ceu{\AwaitExt(e)},e)
      &=\ceu{\Nop}\\[-1\jot]
      %%
      \bcast(\ceu{\AwaitInt(e)},e)
      &=\ceu{\Nop}\\[-1\jot]
      %%
      \bcast(\ceu{\Every{e}\ {p}},e)
      &=\ceu{p;\,\Every{e}\ {p}}\\[-1\jot]
      %%
      \bcast(\ceu{\CanRun(n)},e)
      &=\ceu{\CanRun(n)}\\[-1\jot]
      %%
      \bcast(\ceu{\Fin{p}},e)
      &=\ceu{\Fin{p}}\\[-1\jot]
      %%
      \bcast(\ceu{p;\,q},e)
      &=\ceu{\bcast(p,e);\,q}\\[-1\jot]
      %%
      \bcast(\ceu{p\AtLoop{q}},e)
      &=\ceu{\bcast(p,e)\AtLoop{q}}\\[-1\jot]
      %%
      \bcast(\ceu{p\AtAnd{q}},e)
      &=\ceu{{\bcast(p,e)}\AtAnd{\bcast(q,e)}}\\[-1\jot]
      %%
      \bcast(\ceu{p\AtOr{q}},e)
      &=\ceu{{\bcast(p,e)}\AtOr{\bcast(q,e)}}\\[-1\jot]
      %%
      bcast(\_,e)
      &=\_\enspace
        (\ceu{\Mem},\ceu{\EmitInt},\ceu{\Break},\\[-1\jot]
      &\quad\ceu{\IfElse{}{}},\ceu{\Loop},\ceu{\And},\ceu{\Or},\ceu{\Nop})
      \\[1\jot]
      %%
      %%-
      \shortintertext{\llap{(ii)~}Predicate~$\isblocked$:}
      %%-
      %%
      \isblocked(\ceu{\AwaitExt(e)},n)
      &=\mathit{true}\\[-1\jot]
      %%
      \isblocked(\ceu{\AwaitInt(e)},n)
      &=\mathit{true}\\[-1\jot]
      %%
      \isblocked(\ceu{\Every{e}\ {p}},n)
      &=\mathit{true}\\[-1\jot]
      %%
      \isblocked(\ceu{\CanRun(m)},n)
      &=(n>m)\\[-1\jot]
      %%
      \isblocked(\ceu{\Fin{p}},n)
      &=\mathit{true}\\[-1\jot]
      %%
      \isblocked(\ceu{p;\,q},n)
      &=\isblocked(p,n)\\[-1\jot]
      %%
      \isblocked(\ceu{p\AtLoop{q}},n)
      &=\isblocked(p,n)\\[-1\jot]
      %%
      \isblocked(\ceu{p\AtAnd{q}},n)
      &=\isblocked(p,n)\land\isblocked(q,n)\\[-1\jot]
      %%
      \isblocked(\ceu{p\AtOr{q}},n)
      &=\isblocked(p,n)\land\isblocked(q,n)\\[-1\jot]
      %%
      \isblocked(\_,n)
      &=\mathit{false}\enspace
        (\ceu{\Mem},\ceu{\EmitInt},\ceu{\Break},\\[-1\jot]
      &\quad\ceu{\IfElse{}{}},\ceu{\Loop},\ceu{\And},\ceu{\Or},\ceu{\Nop})
        \\[1\jot]
      %%
      %%-
      \shortintertext{\llap{(iii)~}Function~$\clear$:}
      %%-
      %%
      \clear(\ceu{\AwaitExt(e)})
      &=\ceu{\Nop}\\[-1\jot]
      %%
      \clear(\ceu{\AwaitInt(e)})
      &=\ceu{\Nop}\\[-1\jot]
      %%
      \clear(\ceu{\Every{e}\ p})
      %%
      &=\ceu{\Nop}\\[-1\jot]
      %%
      \clear(\ceu{\CanRun(n)})
      &=\ceu{\Nop}\\[-1\jot]
      %%
      \clear(\ceu{\Fin{p}})
      &=p\\[-1\jot]
      %%
      \clear(\ceu{p;\,q})
      &=\clear(p)\\[-1\jot]
      %%
      \clear(\ceu{p\AtLoop{q}})
      &=\clear(p)\\[-1\jot]
      %%
      \clear(\ceu{p\AtAnd{q}})
      &=\ceu{\clear(p);\,\clear(q)}\\[-1\jot]
      %%
      \clear(\ceu{p\AtOr{q}})
      &=\ceu{\clear(p);\,\clear(q)}\\[-1\jot]
      %%
      \clear(\_)
      &=\xi\enspace
        (\ceu{\Mem},\ceu{\EmitInt},\ceu{\Break},\\[-1\jot]
      &\quad\ceu{\IfElse{}{}},\ceu{\Loop},\ceu{\And},\ceu{\Or},\ceu{\Nop})
    \end{align*}}
\end{gather*}
\vskip-2\belowdisplayskip
\caption{%
  (i)~Function~$\bcast$ awakes awaiting trails matching the event by
  converting~$\ceu{\protect\AwaitExt}$ and~$\ceu{\protect\AwaitInt}$
  to~$\ceu{\protect\Nop}$, and by unwinding $\ceu{\protect\Every}$
  statements.
  %%
  \space(ii)~Predicate~$\isblocked$ is true only if all branches in parallel
  are blocked waiting for events, finalization clauses, or certain
  stack levels.
  %%
  \space(iii)~Function~$\clear$ extracts~$\ceu{\protect\Fin}$ statements in
  parallel and put their bodies in sequence.
  %%
  In~(i), (ii), and~(iii),~``$\_$'' denotes the omitted cases and~``$\xi$''
  denotes the empty string.
  %%
}
\label{fig.bcast}
\label{fig.isblocked}
\label{fig.clear}
\end{figure}

%-
% {\small
% \begin{align*}
%   bcast(e, awaitExt(e)) &= @nop                         \\
%   bcast(e, awaitInt(e)) &= @nop                         \\
%   bcast(e, every~e~p)   &= p;~every~e~p                 \\
%   bcast(e, @canrun(n))  &= @canrun(n)                   \\
%   bcast(e, fin~p)       &= fin~p                        \\
%   bcast(e, p~;~q)       &= bcast(e,p)~;~q               \\
%   bcast(e, p~@loop~q)   &= bcast(e,p)~@loop~q           \\
%   bcast(e, p~@and~q)    &= bcast(e,p)~@and~bcast(e,q)   \\
%   bcast(e, p~@or~q)     &= bcast(e,p)~@or~bcast(e,q)    \\
%   bcast(e, \_)          &= \bot \2 (mem,emitInt,break,if,  \\
%                                  & \5\5 loop,and,or,@nop) %\\
% \end{align*}
% }
%-

%
%-
% {\small
% \begin{align*}
%   isblocked(n, \1 awaitExt(id)) &= true                                   \\
%   isblocked(n, \1 awaitInt(id)) &= true                                   \\
%   isblocked(n, \1 every~e~p)    &= true                                   \\
%   isblocked(n, \1 @canrun(m))   &= (n > m)                                \\
%   isblocked(n, \1 fin~p)        &= true                                   \\
%   isblocked(n, \1 p~;~q)        &= isblocked(n,p)                         \\
%   isblocked(n, \1 p~@loop~q)    &= isblocked(n,p)                         \\
%   isblocked(n, \1 p~@and~q)     &= isblocked(n,p) \wedge isblocked(n,q)   \\
%   isblocked(n, \1 p~@or~q)      &= isblocked(n,p) \wedge isblocked(n,q)   \\
%   isblocked(n, \1 \_)           &= false \2 (mem,emitInt,break,if,        \\
%                                 & \5\5\5\1 loop,and,or,@nop)   %\\
% \end{align*}
% }
%-

%-
% {\small
% \begin{align*}
%   clear( awaitExt(e) ) &= @nop                  \\
%   clear( awaitInt(e) ) &= @nop                  \\
%   clear( every~e~p )   &= @nop                  \\
%   clear( @canrun(n) )  &= @nop                  \\
%   clear( fin~p )       &= p                     \\
%   clear( p~;~q )       &= clear(p)              \\
%   clear( p~@loop~q )   &= clear(p)              \\
%   clear( p~@and~q )    &= clear(p)~;~clear(q)   \\
%   clear( p~@or~q )     &= clear(p)~;~clear(q)   \\
%   clear( \_ )          &= \bot \2 (mem,emitInt,break,if, \\
%                                   & \5\5 loop,and,or,@nop) %\\
% \end{align*}
% }
%-
