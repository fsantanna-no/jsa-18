\section{Detailed Proofs}
\label{sec.appendix}


\lemxranknst*
\begin{proof}
  We proceed by induction on the structure of~$\nst$ derivations.
  Let~$\delta=\<p,n,e>$, $\delta'=\<p',n',e'>$, $\rank(\delta)=\<i,j>$,
  and~$\rank(\delta')=\<i',j'>$.  By the hypothesis of the lemma, there is
  a derivation~$\pi$ such that
  \[
    \pi\Vdash\<p,n,e>\nst\<p',n',e'>\,.
  \]
  By definition of~$\nst$, $e=\nil$ and $n=n'$.  Depending on the structure
  of program~$p$, there are~11 possibilities.  In each one of them we show
  that~$\rank(\delta)\ge\rank(\delta')$.
  \begin{case}
  \item$p=\ceu{\Mem(id)}$.
    %%
    Then~$\pi$ is an instance of~\R{mem}.  Hence $p'=\ceu{\Nop}$
    and~$e'=\nil$.  Thus $\rank(\delta)=\rank(\delta')=\<0,n>$.

  \item$p=\ceu{\EmitInt(e_1)}$.
    %%
    Then~$\pi$ is an instance of~\R{emit-int}.  Hence $p'=\ceu{\CanRun}$
    and~$e'=e_1\ne\nil$.
    Thus
    \[
      {\rank(\delta)={\<1,n>}}>{\<0,n+1>=\rank(\delta')}\,.
    \]

  \item$p=\ceu{\CanRun(n)}$.
    %%
    Then~$\pi$ is an instance of~\R{can-run}.  Hence $p'=\ceu{\Nop}$
    and~$e'=\nil$.  Thus
    \[
      \rank(\delta)=\rank(\delta')=\<0,n>\,.
    \]

  \item$p=\ceu{\IfElse{p}{p_1}{p_2}}$.
    %%
    There are two subcases.
    \begin{case}
    \item\label{lem.x.rank-nst.if-true}$\eval(\ceu{\Mem(\Id)})$.
      %%
      Then~$\pi$ is an instance of~\R{if-true}.  Hence $p'=\ceu{p_1}$
      and~$e'=\nil$.  Thus
      \begin{align*}
        \rank(\delta)&=\<\max\{pot'(p_1),pot'(p_2)\},n>\\
                     &\ge\<\pot'(p_1),n>=\rank(\delta')\,.
      \end{align*}
      %%
    \item$\lnot\eval(\ceu{\Mem(\Id)})$.
      %%
      Similar to Case~\ref{lem.x.rank-nst.if-true}.
    \end{case}

  \item$p=\ceu{p_1;\,p_2}$.
    %%
    There are three subcases.
    \begin{case}
    \item\label{lem.x.rank-nst.seq-nop}$p_1=\ceu{\Nop}$.
      %%
      Then~$\pi$ is an instance of~\R{seq-nop}.
      Hence $p'=p_2$ and~$e'=\nil$.  Thus
      \begin{align*}
        \rank(\delta)&=\<\pot'(p_1)+\pot'(p_2),n>\\
                     &\ge\<pot'(p_2),n>=\rank(\delta')\,.
      \end{align*}
      %%
    \item\label{lem.x.rank-nst.seq-brk}$p_1=\ceu{\Break}$.
      %%
      Then~$\pi$ is an instance of~\R{seq-brk}.
      Hence $p'=p_1$ and~$e'=\nil$.  Thus
      \[
        \rank(\delta)=\rank(\delta')=\<0,n>\,.
      \]
      %%
    \item\label{lem.x.rank-nst.seq-adv}$p_1\ne\ceu{\Nop},\ceu{\Break}$.
      %%
      Then~$\pi$ is an instance of~\R{seq-adv}.  Hence there is a
      derivation~$\pi'$ such that
      \[
        \pi'\Vdash\<p_1,n,\nil>\nst\<p_1',n,e_1'>\,,
      \]
      for some~$p_1'$ and~$e_1'$.  Thus~$p'=p_1';p_2$ and~$e'=e_1'$.  By the
      induction hypothesis,
      \begin{equation}
        \label{lem.x.rank-nst.seq-adv.eq1}
        \rank(\<p_1,n,\nil>)\ge\rank(\<p_1',n,e_1'>)\,.
      \end{equation}
      There are two subcases.
      \begin{case}
      \item$e'=\nil$
        %%
        Then
        \begin{align*}
          \rank(\delta)&=\<\pot'(p_1)+\pot'(p_2),n>\enspace\text{and}\\
          \rank(\delta')&=\<\pot'(p_1')+\pot'(p_2),n>\,.
        \end{align*}
        By~\eqref{lem.x.rank-nst.seq-adv.eq1}, $\pot'(p_1)\ge\pot'(p_1')$.
        Thus
        \[
          \rank(\delta)\ge\rank(\delta')\,.
        \]
        %%
      \item$e'\ne\nil$.
        %%
        Then~$\pi'$ contains one application of~\R{emit-int}, which consumes
        one~$\ceu{\EmitInt(e')}$ statement from~$p_1$ and implies
        $\pot'(p_1)>\pot'(p_1')$.  Thus
        \begin{align*}
          \rank(\delta)&=\<\pot'(p_1)+\pot'(p_2),n>\\
                       &>\<\pot'(p_1')+\pot'(p_2),n+1>=\rank(\delta')\,.
        \end{align*}
      \end{case}
    \end{case}

  \item\label{lem.x.rank-nst.loop-expd}$p=\ceu{\Loop{p_1}}$.
    %%
    Then~$\pi$ is an instance of~\R{loop-expd}.
    Hence $p'=\ceu{p_1\AtLoop{p_1}}$ and~$e'=\nil$.
    %%
    By item~\ref{ass.x.syn-rest.loop} of Assumption~\ref{ass.x.syn-rest},
    all execution paths of~$p_1$ contain at least one occurrence
    of~$\ceu{\Break}$ or~$\ceu{\AwaitExt}$.  Thus, by condition~(\dag) in
    Definition~\ref{def.x.pot},
    \[
      \rank(\delta)=\rank(\delta')=\<\pot'(p_1),n>\,.
    \]

  \item$p=\ceu{{p_1}\AtLoop{p_2}}$.
    %%
    There are three subcases.
    \begin{case}
    \item$p_1=\ceu{\Nop}$.
      %%
      Similar to Case~\ref{lem.x.rank-nst.seq-nop}.
      %%
    \item$p_1=\ceu{\Break}$.
      %%
      Similar to Case~\ref{lem.x.rank-nst.seq-brk}.
      %%
    \item\label{lem.x.rank-nst.loop-adv}$p_1\ne\ceu{\Nop},\ceu{\Break}$.
      %%
      Then~$\pi$ is an instance of~\R{loop-adv}.  Hence there is a
      derivation~$\pi'$ such that
      \[
        \pi'\Vdash\<p_1,n,\nil>\nst\<p_1',n,e_1'>\,,
      \]
      for some~$p_1'$ and~$e_1'$.  Thus~$p'=\ceu{p_1'\AtLoop{\,p_2}}$
      and~$e'=e_1'$.
      %%
      % By the induction hypothesis,
      % \begin{equation}
      %   \label{lem.x.rank-nst.loop-adv-eq1}
      %   \rank(\<p_1,n,\nil>)\ge\rank(\<p_1',n,e_1'>)\,.
      % \end{equation}
      %%
      There are two subcases.
      \begin{case}
      \item$\pot'(p)=\pot'(p_1)$.
        %%
        Then every execution path of~$p_1$ contains a~$\ceu{\Break}$
        or~$\ceu{\AwaitExt}$ statement.  A single~$\nst$ cannot terminate
        the loop, since~$p_1\ne\ceu{\Break}$, nor can it consume
        an~$\ceu{\AwaitExt}$, which means that all execution paths in~$p_1'$
        still contain a~$\ceu{\Break}$ or~$\ceu{\AwaitExt}$.
        Hence $\pot'(p')=\pot'(p_1')$.  The rest of this proof is similar to
        that of Case~\ref{lem.x.rank-nst.seq-adv}.
        %%
      \item$\pot'(p)=\pot'(p_1)+\pot'(p_2)$.
        %%
        Then some execution path in~$p_1$ does not contain a~$\ceu{\Break}$
        or~$\ceu{\AwaitExt}$ statement.  Since~$p_1\ne\ceu{\Nop}$, a
        single~$\nst$ cannot restart the loop, which means that~$p_1'$ still
        contain some execution path in which a~$\ceu{\Break}$
        or~$\ceu{\AwaitExt}$ does not occur.
        Hence $\pot'(p')=\pot'(p_1')+\pot'(p_2)$.  The rest of this proof is
        similar to that of Case~\ref{lem.x.rank-nst.seq-adv}.
      \end{case}
    \end{case}

  \item\label{lem.x.rank-nst.and-expd}$p=\ceu{{p_1}\And{p_2}}$.
    %%
    Then~$\pi$ is an instance of~\R{and-expd}.
    Hence $p'=\ceu{{p_1}\AtAnd(\CanRun(n);\,p_2)}$ and~$e'=\nil$.
    Thus
    \[
      \rank(\delta)=\rank(\delta')=\<\pot'(p_1)+\pot'(p_2),n>\,.
    \]

  \item\label{lem.x.rank-nst.and}$p=\ceu{{p_1}\AtAnd{p_2}}$.
    %%
    There are two subcases.
    stack level~$n$.
    \begin{case}
    \item\label{lem.x.rank-nst.and1}$\lnot\isblocked(p_1,n)$.
      %%
      There are three subcases.
      \begin{case}
      \item\label{lem.x.rank-nst.and-nop1}$p_1=\ceu{\Nop}$.
        %%
        Then~$\pi$ is an instance of~\R{and-nop1}.  Hence $p'=p_2$
        and~$e'=\nil$.  Thus
        \[
          \rank(\delta)=\rank(\delta')=\<0+\pot'(p_2),n>\,.
        \]
        %%
      \item\label{lem.x.rank-nst.and-brk1}
        $p_1=\ceu{\Break}$.
        %%
        Then~$\pi$ is an instance of~\R{and-brk1}.
        Hence $p'=\ceu{\clear(p_2);\,\Break}$ and~$e'=\nil$.  By
        item~\ref{ass.x.syn-rest.fin} of Assumption~\ref{ass.x.syn-rest} and
        by the definition of~$\clear$, $\clear(p_2)$ does not
        contain~$\ceu{\EmitInt}$ statements.  Thus
        \[
          \rank(\delta)=\rank(\delta')=\<0,n>\,.
        \]
        %%
      \item\label{lem.x.rank-nst.and-adv1}$p_1\ne\ceu{\Nop},\ceu{\Break}$.
        %%
        Then~$\pi$ is an instance of~\R{and-adv1}.  As~$p_1\ne\ceu{\Break}$
        and~$p_2\ne\ceu{\Break}$ (otherwise~\R{and-brk2} would have taken
        precedence), the rest of this proof is similar to that of
        Case~\ref{lem.x.rank-nst.seq-adv}.
      \end{case}
      %%
    \item$\isblocked(p_1,n)$.
      %%
      Similar to Case~\ref{lem.x.rank-nst.and1}
    \end{case}

  \item$p=\ceu{{p_1}\Or{p_2}}$.
    %%
    Then~$\pi$ is an instance of~\R{or-expd}.
    Hence $p'=\ceu{{p_1}\AtOr(\CanRun(n);\,p_2)}$ and~$e'=\nil$.  Thus
    \[
      \rank(\delta)=\rank(\delta')=\<\pot'(p_1)+\pot'(p_2),n>\,.
    \]

  \item$p=\ceu{{p_1}\AtOr{p_2}}$.
    %%
    There are two subcases.
    \begin{case}
    \item\label{lem.x.rank-nst.or1}$\lnot\isblocked(p_1,n)$.
      %%
      There are three subcases.
      \begin{case}
      \item$p_1=\ceu{\Nop}$.
        %%
        Then~$\pi$ is an instance of~\R{or-nop1}.  Hence $p'=\clear(p_2)$
        and~$e'=\nil$.  By item~\ref{ass.x.syn-rest.fin} of
        Assumption~\ref{ass.x.syn-rest} and by the definition of~$\clear$,
        $p'$ does not contain~$\ceu{\EmitInt}$ statements.  Thus
        \[
          \rank(\delta)=\rank(\delta')=\<0,n>\,.
        \]
        %%
      \item$p_1=\ceu{\Break}$.
        %%
        Similar to Case~\ref{lem.x.rank-nst.and-brk1}.
        %%
      \item$p_1\ne\ceu{\Nop},\ceu{\Break}$.
        %%
        Similar to Case~\ref{lem.x.rank-nst.and-adv1}.
      \end{case}
    \item$\isblocked(p_1,n)$.
      %%
      Similar to Case~\ref{lem.x.rank-nst.or1}.\qedhere
    \end{case}
  \end{case}
\end{proof}


\thmxranknstx*
\begin{proof}
  If~$\delta\nst[*]\delta'$ then~$\delta\nst[i]\delta'$, for some~$i$.  We
  proceed by induction on~$i$.
  %%
  The theorem is trivially true for~$i=0$ and follows directly from
  Lemma~\ref{lem.x.rank-nst} for~$i=1$.  Suppose
  $\delta\nst[1]\delta_1'\nst[i-1]\delta'$, for some~$i>1$ and~$\delta_1'$.
  Thus, by Lemma~\ref{lem.x.rank-nst} and by the induction hypothesis,
  \[
    \rank(\delta)\ge\rank(\delta_1')\ge\rank(\delta')\,.\qedhere
  \]
\end{proof}

%\balance

\thmxterm*
\begin{proof}
  By lexicographic induction on~$rank(\delta)$.  Let~$\delta=\<p,n,e>$ and
  $\rank(\delta)=\<i,j>$.

  \vskip\baselineskip
  \noindent\emph{Basis}.
  If~$\<i,j>=\<0,0>$ then~$\delta$ cannot be advanced by~$\out$, as~$j=0$
  implies~$e=\nil$ and~$n=0$ (neither~\R{push} nor~\R{pop} can be applied).
  There are two possibilities:  either~$\delta$ is nested irreducible or it
  is not.  In the first case, the theorem is trivially true,
  as~$\delta\nst[0]\delta_\Hnst$.  Suppose~$\delta$ is not
  nested irreducible.  Then, by Theorem~\ref{thm.x.term-nst-*},
  $\delta\nst[*]\delta'_\Hnst$, for some~$\delta'_\Hnst$.  By
  Theorem~\ref{thm.x.rank-nst-*},
  \[
    \<i,j>=\<0,0>\ge\rank(\delta')\,,
  \]
  which implies~$\rank(\delta')=\<0,0>$.

  \vskip\baselineskip
  \noindent\emph{Induction}.
  Let~$\<i,j>\ne\<0,0>$.
  %%
  There are two subcases.
  \begin{case}
  \item\label{thm.x.term.Hnst}$\delta$~is nested-irreducible.
    %%
    There are two subcases.
    \begin{case}
    \item\label{thm.x.term.Hnst-j0}$j=0$.
      %%
      By Definition~\ref{def.x.H}, $\delta_\#$.
      Thus~$\delta\trans[0]\delta_\#$.
      %%
    \item\label{thm.x.term.Hnst-j>0}$j>0$.
      %%
      There are two subcases.
      event.
      \begin{case}
      \item\label{thm.x.term.Hnst-j>0-nonnil}$e\ne\nil$.
        %%
        Then, by~\R{push} and by Theorem~\ref{thm.x.term-nst-*}, there
        are~$\delta_1'$ and~$\delta'_\Hnst=\<p',n+1,e'>$ such that
        \[
          \delta\outpush\delta_1'\nst[*]\delta'_\Hnst\,.
        \]
        Thus, by item~\ref{lem.x.rank-out-push} of
        Lemma~\ref{lem.x.rank-out} and by Theorem~\ref{thm.x.rank-nst-*},
        \begin{align*}
          \rank(\delta)=\rank(\delta_1')&=\<i,j>\\
                                        &\ge\rank(\delta')=\<i',j'>\,.
        \end{align*}
        If~$e'=\nil$, then $i=i'$ and~$j=j'$, and the rest of this proof
        is similar to that of Case~\ref{thm.x.term.Hnst-j>0-nil}.
        Otherwise, if~$e'\ne\nil$ then $i>i'$, since
        an~$\ceu{\EmitInt(e')}$ was consumed by the nested transitions.
        Thus,
        \[
          \rank(\delta)>\rank(\delta')\,.
        \]
        By the induction hypothesis, $\delta'\trans[*]\delta''_\#$, for
        some~$\delta''_\#$.  Therefore, $\delta\trans[*]\delta''_\#$.
        %%
      \item\label{thm.x.term.Hnst-j>0-nil}$e=\nil$.
        %%
        Then, since~$j>0$, $\delta\outpop\delta'$, for some~$\delta''$.
        By item~\ref{lem.x.rank-out-pop} of Lemma~\ref{lem.x.rank-out},
        \[
          \rank(\delta)>\rank(\delta')\,.
        \]
        Hence by the induction hypothesis, there is a~$\delta''_\#$ such
        that~$\delta'\trans[*]\delta''_\#$.
        Therefore, $\delta\trans[*]\delta''_\#$.
      \end{case}
    \end{case}
    %%
  \item$\delta$~is not nested-irreducible.
    %%
    Then~$e=\nil$ and, by Theorems~\ref{thm.x.term-nst-*}
    and~\ref{thm.x.rank-nst-*}, there is a~$\delta'_\Hnst$ such
    that~$\delta\nst[*]\delta'_\Hnst$
    with~$\rank(\delta)\ge\rank(\delta'_\Hnst)$.  The rest of this proof
    is similar to that of Case~\ref{thm.x.term.Hnst}.\qedhere
  \end{case}
\end{proof}
